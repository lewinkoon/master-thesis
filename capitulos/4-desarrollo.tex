En este capítulo se explica cómo ha sido la dinámica del trabajo a la hora de analizar las tareas administrativas del centro además de desarrollar en detalle todos los procesos que se llevan a cabo en la oficina.

\section{Distribución de la plantilla}

El equipo administrativo del centro de salud está compuesto por siete miembros: dos puestos de interior, cuatro en ventanilla y un jefe de grupo.

\section{Procesos administrativos}

En este apartado se describen los distintos procesos que existen en la sección administrativa. 



\subsection{Gestión de citas previas}

% Un paciente puede solicitar una cita con un profesional sanitario de atención primaria por dos vías: en el propio centro de salud o a través de la aplicación de SACYL Conecta.

% Si el usuario decide solicitar cita previa vía online la realización de la tarea es diferida, no tiene por qué realizarse en el momento. De esta forma, al final de la jornada o cuando el trabajo lo crea oportuno, se revisan las peticiones de citas y se tramitan en el programa Medora.

\subsubsection{Consultas simtron}

Desde septiembre de 2016, se citan en la consulta de enfermería desde \textit{Medora}.
A excepción de los pacientes procedentes de residencias geriátricas, que se citan en el puesto interior de laboratorio.
Una vez se reciben los resultados del \textit{Simtrom}, se envían por fax a las distintas residencias geriátricas.

Del mismo modo, enfermería se encargará de obtener e imprimir los resultados de la prueba en el puesto interior.

En algunas ocasiones, a primera hora de la mañana una de las residencias envía vía fax la nota de Sintrom de alguno de los pacientes.
En este caso, se citará en el mismo día en la consulta de enfermería y se colocará el documento en el casillero correspondiente al enfermero.

% \subsection{Agendas}

% EL proceso de realización de agendas consiste en la asignación de tareas sanitarias en la distintas franjas de tiempo de la jornada laboral. Las agendas están orientadas para el personal sanitario que pasan consulta o realizan algún tipo de prueba al paciente: médicos, enfermeras, fisioterapeutas, trabajadores sociales, etc.

% Poniendo de ejemplo un médico facultativo, 

% \subsection{Gestión de la tarjeta sanitaria}

% Tanto la creación de una nueva tarjeta sanitaria como la modificación de algunos de sus parámetros es otro de los procesos más demandados por la población. Ambas operaciones se realizan a través del programa \textit{Tarjeta Sanitaria} que actúa como base de datos de las datos identificativos de los usuarios asignados a la Zona Básica de Salud a la que pertenece el centro, en este caso Laguna de Duero.

% Se pueden editar multitud de campos, pero las operaciones más demandadas son las siguientes:

% \begin{enumerate}
%     \item Creación de tarjeta sanitaria
%     \item Cambio de médico de familia
%     \item Cambio de consultorio o de centro de salud (conlleva cambio de médico de familia)
% \end{enumerate}

\subsection{Cargos a terceros}

Por lo general, los pacientes que acuden al centro de salud tienen acceso gratuito a la cartera de servicios de Atención Primaria, sin embargo, hay ciertos casos en los que el Sistema Público de Salud no asume el coste y que por tanto se debe de cobrar al paciente.
Estos casos pueden ser:

\subsubsection{Convenios internacionales}

En este caso de debe rellenar el formulario \textit{H-111}, indicando diagnóstico, fecha de asistencia y firma del paciente.
Es muy importante presentar una fotocopia de la tarjeta sanitaria europea o un documento equivalente.

En el caso de que no se presente, se debe anotar el número de teléfono y el correo electrónico para poder contactar con el paciente desde la sección de ``Cargos a terceros'' del Hospital Universitario Rio Hortega.

También es necesario que el paciente firme la nota informativa ya que es el documento requerido para la protección de datos y el compromiso de hacerse cargo del importe del gasto.

\subsubsection{Accidentes de tráfico}

En el caso de un accidente de tráfico es necesario que el paciente rellene la ficha de asistencia sanitaria, firme la nota informativa y complete la solicitud de datos de accidente de tráfico.
Este último documento debe concordar con los datos presentes en el atestado del accidente.

\subsubsection{Accidentes de trabajo}

En este caso tan solo es necesario rellenar la ficha de asistencia sanitaria y firmar la nota informativa.

\subsubsection{Espectáculos taurinos}

Se rellena la ficha de asistencia sanitaria junto con la firma de la nota interior.
En este caso, se debe de completar el documento ``Lesionados en espectáculos taurinos'' especificando el lugar donde se realiza el espectáculo y quién es el responsable del mismo (Ayuntamiento, asociaciones taurinas, etc).

\subsubsection{Agresiones}

En este caso tan solo es necesario rellenar la ficha de asistencia sanitaria y firmar la nota informativa.

\subsubsection{Accidentes deportivos o de caza}

Además de la ficha de asistencia sanitaria y la nota informativa, el paciente debe presentar una fotocopia de su licencia federativa junto con la información del seguro que cubre el siniestro, especificando el número de póliza y el número de siniestro.

\subsubsection{Seguros privados}

Se rellena la ficha de asistencia sanitaria junto con la firma de la nota informativa.
Se debe presentar también una copia del parte de accidente, la petición de asistencia o cualquier otro documento de acepctación del gasto que haya sido expedido por la empresa, la compañía de seguros o la mutua.
En el caso de que sea un paciente desplazado, se debe presentar la fotocopia de la tarjeta de mutualista.

\subsubsection{Particulares}

Se completa la ficha de asistencia sanitaria y se firma la nota informativa. Este último documento es muy importante ya que hace constar que el paciente se responsabiliza del importe del cargo.

\subsection{Reclamaciones}

El proceso de reclamaciones se inicia, generalmente, cuando un usuario se acerca a la ventanilla de admisión solicitando poner una reclamación.
El usuario deberá identificarse con su documento de identidad y el personal administrativo cumplimentará la hoja de reclamación a tal efecto.
Una vez tomados los datos, el administrativo entrega al usuario la correspondiente hoja de reclamaciones para que la rellene con su demanda.
Una vez entregada la hoja de vuelta por el usuario, se registra de entrada y se entrega la hoja amarilla al paciente mientras que las dos hojas restantes se quedan en administración.

Tras esta primera fase, si el objeto de la reclamación está relacionado con un trámite administrativo es el propio jefe de grupo quien contesta la reclamación. Si no es el caso, es el propio jefe de grupo quien entrega la reclamación al responsable correspondiente.

Una vez la reclamación ha sido contestada por el responsable, se envía una de las hojas a la Gerencia de Atención Primaria a través de correo físico ya que son ellos quienes se encargan de resolver formalmente la reclamación y hacérsela llegar al usuario a su domicilio.

Cuando la Gerencia de Atención Primaria resuelve una reclamación, además de enviar la resolución al usuario, se envía una copia al centro de salud donde se interpuso la demanda para que tengan constancia de que fue resuelta en tiempo y forma.
Este último paso es importante ya que hay casos en que los usuarios acuden al centro de salud preguntando por el estado de su reclamación.

\subsection{Absentismo}

La gestión de permisos retribuidos de los profesionales del centro de salud también recae en la sección administrativa.
Cuando un profesional quiere solicitar un permiso retribuido, debe completar el la la petición correspondiente.
Tras completar la petición se debe entregar al responsable, en el caso de médico se entre al coordinador del centro, para recibir la aprobación.
Si los profesionales son facultativos, por lo general, se deja la petición en el casillero habilitado del coordinador para que lo recoja en cuanto pueda.

Tras la aprobación del responsable, es el administrativo quien se encarga de ensobrar la petición y enviarla por correo físico a la Gerencia de Atención Primaria para su autorización.
Siempre se debe de adjuntar el original con las dos firmas, la del responsable y la del solicitante.
Se espera a recibir de vuelta la petición autorizada por la Gerencia y se realiza una fotocopia.
El original se entrega al profesional mientras que la fotocopia se guarda en la carpeta administrativa del mismo profesional, en la parte delantera de la ficha.

En numerosas ocasiones, debido a la lentitud del proceso provocada por el movimiento físico de papeles por correo, se da la situación de que el profesional disfruta del permiso antes, incluso, de recibir la autorización desde la Gerencia.
Más adelante, se tratará una de las soluciones propuestas para resolver este cuello de botella.

\subsection{Tareas generales}

\subsubsection{Informar al paciente}

Una de las funciones de los puestos administrativos de ventanilla es la de informar a los paciente y entregar justificantes de diversos tipos. Aunque las solicitudes en este proceso son muy diversas, en general se pueden distinguir tres tipos:

\begin{itemize}
    \item Utilización de la aplicación de SACYL Conecta
    \item Impresión del parte de baja laboral
    \item Impresión de justificante COVID
\end{itemize}

Todas estas tareas en general son sencillas y consumen poco tiempo al administrativo.

\subsubsection{Enfermedades de declaración obligatoria}

Las enfermedades de declaración obligatoria (EDO) o enfermedades de notificación obligatoria (ENO) son aquellas enfermedades que se consideran de gran importancia para la salud pública. Los médicos están obligados a notificar al centro de salud pública correspondiente por ser de especial importancia para la comunidad.

La notificación permite recoger datos estadísticos que muestren la frecuencia con la cual ocurre la enfermedad, lo cual, a su vez, ayuda a los investigadores a identificar las tendencias de la enfermedad y a rastrear sus brotes. Esto puede ayudar a controlar brotes futuros.

Todos las comunidades tienen un listado de las enfermedades de notificación obligatoria. Es responsabilidad del proveedor de atención médica, no del paciente, reportar los casos de estas enfermedades.

Para notificar la enfermedad se rellena el modelo C todos los lunes de cada semana.
Si existen hoja de declaración individual, se hace fotocopia de cada una de ellas.
Posteriormente, se registra en el programa \textit{Medora} el informe semanal y se imprimen dos copias.

% citar orden SAN/2128/2006, de 27 de diciembre

\subsubsection{Recetas con visado de inspección}

El paciente puede entregar en ventanilla recetas que necesiten un visado por inspección.
Se guardan en el sobre de ``recetas-inspección'' del puesto interior y se indica al usuario que pase por el centro en ocho días naturales para recogerla.

Posteriormente se envían al a Inspección Médica por correo físico, registrando siempre la fecha de envío.
Cuando vengan selladas, se incluyen por orden alfabético de la carpeta fuelle ``Recetas visadas'' para tenerlas a mano cuando el paciente venga a recogerlas.

\subsubsection{Estadísticas de ginecología}

Diariamente, se recoge de la consulta de ginecología las estadísticas diarias.
Se contabilizan las primeras consultas de ginecología y obstetricia, junto con las revisiones de ambas.
Posteriormente, se registran los datos en el programa \textit{Medora}.

\subsubsection{Reubicación de pacientes}

Por motivos de cierre o bloqueo de las agendas de los profesionales, puede suceder que haya que reprogramar las citas ya dadas a los pacientes.
Se trata de llamar al paciente para cambiar la cita previa. Si esto sucede es importante indicar en el programa \textit{Medora} que la cita está reubicada en el campo de observaciones.

Si la cita ha sido programadas por enfermería, son ellos mismos quienes realizan la reubicación.

\subsubsection{Diseño de cartelería}

Según la programación, los viernes de cada semana, se crean los carteles correspondientes para conocer en qué consulta serán atendidos los pacientes cuyos profesionales estén ausentes.

Concretamente, se realizarán para los profesionales médicos, enfermería y pediatras.

\subsubsection{Atención telefónica}

En ocasiones, se reciben llamadas de urgencias que se deben de atender en primer término. Se transfieren al facultativo presente en el centro en el momento de la llamada.

También pueden llegar llamadas del servicio de emergencias del 112. Este tipo de llamadas no se deben de transferir al facultativo.

\section{Análisis causa raíz}