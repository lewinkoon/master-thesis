En este capítulo se explica cómo ha sido la dinámica del trabajo a la hora de analizar las tareas administrativas del centro además de desarrollar en detalle todos los procesos que se llevan a cabo en la oficina. 

\section{Grupo de trabajo}

Para conocer en detalle la sección administrativa del centro se organizaron reuniones semanales con un grupo pequeño de trabajadores a modo de muestra que conocieran bien su puesto de trabajo:

\begin{itemize}
    \item Jefe de grupo (Asunción)
    \item Administrativo de ventanilla (Ángeles)
    \item Trabajador puesto interior (Óscar)
\end{itemize}

Se realizaron un total de seis reuniones a lo largo de dos meses en las que se trataron todos los procesos además de analizar la problemática de cada uno.

\subsection{Reunión 1: toma de contacto}

En esta primera reunión asistieron los dos subdirectores de gestión, Diego Vecillas y María Mohíno. Fue una presentación corta en la que se explicó al grupo de trabajo la necesidad de realizar una estandarización de las tareas administrativas de la sección del centro con el fin de analizar las cargas de trabajo reales. Además, se propuso de analizar mediante la técnica de los cinco porqués las causas raíz de los problemas administrativos.

\subsection{Reunión 2: citas previas}

Dado que la gestión de citas principales, y consecuentemente más tiempo consumen, fue el primer proceso que se trató. Se explicó con detalle los distintos tipos de citas: médico de familia, enfermera, trabajador social, ginecología, extracción y radiología. Además de explicar el proceso, se entregó una copia a modo de ejemplo de los volantes de extracciones y de radiología en los cuales el facultativo especifica qué pruebas concretas se deben de realizar al paciente.

\subsection{Reunión 3: }

\section{Modelado de los procesos}

En este apartado se describen los distintos procesos que existen en la sección administrativa. 

\subsection{Agendas}

EL proceso de realización de agendas consiste en la asignación de tareas sanitarias en la distintas franjas de tiempo de la jornada laboral. Las agendas están orientadas para el personal sanitario que pasan consulta o realizan algún tipo de prueba al paciente: médicos, enfermeras, fisioterapeutas, trabajadores sociales, etc.

Poniendo de ejemplo un médico facultativo, 

\subsection{Cargos a terceros}

En general, la mayoría de pacientes tienen acceso gratuito a la cartera de servicios de Atención Primaria, sin embargo, hay ciertos casos en los que el Sistema Público de Salud no asume el coste y que por tanto se debe de cobrar al paciente. Estos casos pueden ser:

\begin{enumerate}
    \item Accidentes de trabajo y enfermedades profesionales a cargo de las mutuas de accidente de trabajo, del Instituto Nacional de Seguridad Social o del Instituto Social de la Marina
    \item Asegurados de empresas colaboradoras en la asistencia sanitaria del sistema de Seguridad Social
    \item Asegurados y beneficiarios de Mutualidades ( Muface, Mugeju, Isfas) que han optado a recibir asistencia sanitaria por una aseguradora privada.
    \item Accidentes de tráfico
    \item Seguros obligatorios: de deportistas federados y profesionales, de viajeros, de caza y cualquier otro seguro obligatorio
    \item Convenios y conciertos con organismos o entidades
    \item Seguro escolar
    \item Asistencias de ciudadanos extranjeros
    \item Accidentes acaecidos con ocasión de eventos festivos, actividades recreativas y espectáculos públicos
\end{enumerate}

% Basarse en la nota interior de cargos a terceros. Exponer los distintos casos y especificar las documentos necesarios que se realizan en cada caso.

\subsection{Gestión de citas}

Un paciente puede solicitar una cita con un profesional sanitario de atención primaria por dos vías: en el propio centro de salud o a través de la aplicación de SACYL Conecta.

Si el usuario decide solicitar cita previa vía online la realización de la tarea es diferida, no tiene por qué realizarse en el momento. De esta forma, al final de la jornada o cuando el trabajo lo crea oportuno, se revisan las peticiones de citas y se tramitan en el programa Medora.

\subsection{Gestión de la tarjeta sanitaria}

Tanto la creación de una nueva tarjeta sanitaria como la modificación de algunos de sus parámetros es otro de los procesos más demandados por la población. Ambas operaciones se realizan a través del programa \textit{Tarjeta Sanitaria} que actúa como base de datos de las datos identificativos de los usuarios asignados a la Zona Básica de Salud a la que pertenece el centro, en este caso Laguna de Duero.

Se pueden editar multitud de campos, pero las operaciones más demandadas son las siguientes:

\begin{enumerate}
    \item Creación de tarjeta sanitaria
    \item Cambio de médico de familia
    \item Cambio de consultorio o de centro de salud (conlleva cambio de médico de familia)
\end{enumerate}

\subsection{Información general}

Una de las funciones de los puestos administrativos de ventanilla es la de informar a los paciente y entregar justificantes de diversos tipos. Aunque las solicitudes en este proceso son muy diversas, en general se pueden distinguir tres tipos:

\begin{itemize}
    \item Utilización de la aplicación de SACYL Conecta
    \item Impresión del parte de baja laboral
    \item Impresión de justificante COVID
\end{itemize}

\subsection{Tramitación de permisos retribuidos}

Otro de los procesos que más tiempo y esfuerzo administrativo requiere son las tramitaciones de permisos retribuidos del personal del centro. Esto incluye a médicos, enfermeras, auxiliares, técnicos de rayos, celadores y a los propios auxiliares administrativos. Más adelante, en el capítulo de propuestas de mejora, se hablará de la implantación del programa AIDA de gestión de personal que elimina todos los trámites en papel y el envío físico de los justificantes a los distintos aprobadores.

Sin embargo, a día de hoy, el flujo de información sigue siendo manual.

\begin{enumerate}
    \item \textbf{Recepción de solicitudes de permiso}. En cualquier momento del día se pueden recibir solicitudes de permisos por parte del personal. Pueden estar o no aprobadas por el jefe de servicio correspondiente (coordinador médico o la supervisora de enfermería). 
    \item Envío de 
\end{enumerate}

\subsection{Reclamaciones}

\section{Análisis causa raíz}