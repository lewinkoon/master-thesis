En este capítulo se explica cómo ha sido la dinámica del trabajo a la hora de analizar las tareas administrativas del centro además de desarrollar en detalle todos los procesos que se llevan a cabo en la oficina.

\section{Distribución de la plantilla}

El equipo administrativo del centro de salud está 

\section{Procesos administrativos}

En este apartado se describen los distintos procesos que existen en la sección administrativa. 

% \subsection{Agendas}

% EL proceso de realización de agendas consiste en la asignación de tareas sanitarias en la distintas franjas de tiempo de la jornada laboral. Las agendas están orientadas para el personal sanitario que pasan consulta o realizan algún tipo de prueba al paciente: médicos, enfermeras, fisioterapeutas, trabajadores sociales, etc.

% Poniendo de ejemplo un médico facultativo, 

\subsection{Cargos a terceros}

Por lo general, los pacientes que acuden al centro de salud tienen acceso gratuito a la cartera de servicios de Atención Primaria, sin embargo, hay ciertos casos en los que el Sistema Público de Salud no asume el coste y que por tanto se debe de cobrar al paciente.
Estos casos pueden ser:

\subsubsection{Convenios internacionales}

En este caso de debe rellenar el formulario \textit{H-111}, indicando diagnóstico, fecha de asistencia y firma del paciente.
Es muy importante presentar una fotocopia de la tarjeta sanitaria europea o un documento equivalente.

En el caso de que no se presente, se debe anotar el número de teléfono y el correo electrónico para poder contactar con el paciente desde la sección de ``Cargos a terceros'' del Hospital Universitario Rio Hortega.

También es necesario que el paciente firme la nota informativa ya que es el documento requerido para la protección de datos y el compromiso de hacerse cargo del importe del gasto.

\subsubsection{Accidentes de tráfico}

En el caso de un accidente de tráfico es necesario que el paciente rellene la ficha de asistencia sanitaria, firme la nota informativa y complete la solicitud de datos de accidente de tráfico.
Este último documento debe concordar con los datos presentes en el atestado del accidente.

\subsubsection{Accidentes de trabajo}

En este caso tan solo es necesario rellenar la ficha de asistencia sanitaria y firmar la nota informativa.

\subsubsection{Espectáculos taurinos}

Se rellena la ficha de asistencia sanitaria junto con la firma de la nota interior.
En este caso, se debe de completar el documento ``Lesionados en espectáculos taurinos'' especificando el lugar donde se realiza el espectáculo y quién es el responsable del mismo (Ayuntamiento, asociaciones taurinas, etc).

\subsubsection{Agresiones}

En este caso tan solo es necesario rellenar la ficha de asistencia sanitaria y firmar la nota informativa.

\subsubsection{Accidentes deportivos o de caza}

Además de la ficha de asistencia sanitaria y la nota informativa, el paciente debe presentar una fotocopia de su licencia federativa junto con la información del seguro que cubre el siniestro, especificando el número de póliza y el número de siniestro.

\subsubsection{Seguros privados}

Se rellena la ficha de asistencia sanitaria junto con la firma de la nota informativa.
Se debe presentar también una copia del parte de accidente, la petición de asistencia o cualquier otro documento de acepctación del gasto que haya sido expedido por la empresa, la compañía de seguros o la mutua.
En el caso de que sea un paciente desplazado, se debe presentar la fotocopia de la tarjeta de mutualista.

\subsubsection{Particulares}

Se completa la ficha de asistencia sanitaria y se firma la nota informativa. Este último documento es muy importante ya que hace constar que el paciente se responsabiliza del importe del cargo.

\subsection{Reclamaciones}

Se entregarán al usuario siempre que lo solicite. El usuario deberá identificarse con su documento de identidad y el personal administrativo cumplimentará la hoja elaborada a tal efecto, donde se encuentran las hojas de reclamación.

Una vez entregada por el usuario, se registra de entrada y se entrega la hoja amarilla al paciente. Las dos hojas restantes se quedan en administración.

\subsection{Gestión de citas}

Un paciente puede solicitar una cita con un profesional sanitario de atención primaria por dos vías: en el propio centro de salud o a través de la aplicación de SACYL Conecta.

Si el usuario decide solicitar cita previa vía online la realización de la tarea es diferida, no tiene por qué realizarse en el momento. De esta forma, al final de la jornada o cuando el trabajo lo crea oportuno, se revisan las peticiones de citas y se tramitan en el programa Medora.

\subsection{Gestión de la tarjeta sanitaria}

Tanto la creación de una nueva tarjeta sanitaria como la modificación de algunos de sus parámetros es otro de los procesos más demandados por la población. Ambas operaciones se realizan a través del programa \textit{Tarjeta Sanitaria} que actúa como base de datos de las datos identificativos de los usuarios asignados a la Zona Básica de Salud a la que pertenece el centro, en este caso Laguna de Duero.

Se pueden editar multitud de campos, pero las operaciones más demandadas son las siguientes:

\begin{enumerate}
    \item Creación de tarjeta sanitaria
    \item Cambio de médico de familia
    \item Cambio de consultorio o de centro de salud (conlleva cambio de médico de familia)
\end{enumerate}

\subsection{Información general}

Una de las funciones de los puestos administrativos de ventanilla es la de informar a los paciente y entregar justificantes de diversos tipos. Aunque las solicitudes en este proceso son muy diversas, en general se pueden distinguir tres tipos:

\begin{itemize}
    \item Utilización de la aplicación de SACYL Conecta
    \item Impresión del parte de baja laboral
    \item Impresión de justificante COVID
\end{itemize}

\subsection{Tramitación de permisos retribuidos}

Otro de los procesos que más tiempo y esfuerzo administrativo requiere son las tramitaciones de permisos retribuidos del personal del centro. Esto incluye a médicos, enfermeras, auxiliares, técnicos de rayos, celadores y a los propios auxiliares administrativos. Más adelante, en el capítulo de propuestas de mejora, se hablará de la implantación del programa AIDA de gestión de personal que elimina todos los trámites en papel y el envío físico de los justificantes a los distintos aprobadores.

Sin embargo, a día de hoy, el flujo de información sigue siendo manual.

\begin{enumerate}
    \item \textbf{Recepción de solicitudes de permiso}. En cualquier momento del día se pueden recibir solicitudes de permisos por parte del personal. Pueden estar o no aprobadas por el jefe de servicio correspondiente (coordinador médico o la supervisora de enfermería). 
    \item Envío de 
\end{enumerate}

\section{Análisis causa raíz}