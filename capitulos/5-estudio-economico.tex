En este capítulo se muestran la planificación y prespuesto económico teórico del proyecto realizado tanto a nivel de personal como de material.

\section{Fases del proyecto}

Para conocer en detalle la sección administrativa del centro se organizaron reuniones semanales con un grupo pequeño de trabajadores a modo de muestra que conocieran bien su puesto de trabajo:

\begin{itemize}
    \item Jefe de grupo (Asunción)
    \item Administrativo de ventanilla (Ángeles)
    \item Trabajador puesto interior (Óscar)
\end{itemize}

Se realizaron un total de seis reuniones a lo largo de dos meses en las que se trataron todos los procesos además de analizar la problemática de cada uno.

\subsection{Reunión 1: toma de contacto}

En esta primera reunión asistieron los dos subdirectores de gestión, Diego Vecillas y María Mohíno. Fue una presentación corta en la que se explicó al grupo de trabajo la necesidad de realizar una estandarización de las tareas administrativas de la sección del centro con el fin de analizar las cargas de trabajo reales. Además, se propuso de analizar mediante la técnica de los cinco porqués las causas raíz de los problemas administrativos.

\subsection{Reunión 2: citas previas}

Dado que la gestión de citas principales, y consecuentemente más tiempo consumen, fue el primer proceso que se trató. Se explicó con detalle los distintos tipos de citas: médico de familia, enfermera, trabajador social, ginecología, extracción y radiología. Además de explicar el proceso, se entregó una copia a modo de ejemplo de los volantes de extracciones y de radiología en los cuales el facultativo especifica qué pruebas concretas se deben de realizar al paciente.

\subsection{Reunión 3: }

\section{Coste económico}