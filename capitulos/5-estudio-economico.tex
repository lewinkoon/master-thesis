En este capítulo se muestran la planificación y prespuesto económico teórico del proyecto realizado tanto a nivel de personal como de material.

\section{Fases del proyecto}

Conocer las fases del proyecto es necesario para estimar los costes económicos asociados a cada fase.
A continuación, se describen las fases en las que se ha desarollado el proyecto:

\subsection{Propuesta}

El proyecto comienza con la propuesta por parte de la Subdirección de Gestión del Área de Salud Valladolid Oeste. En esta fase se exponen los objetivos que se quieren alcanzar y además de propuestas de cara a la metodología a seguir.

\subsection{Preparación}

Antes de formar al grupo de trabajo del centro de salud, es necesario documentarse en la aplicación de técnicas lean y de mejora continua en entornos de oficina y sanitarios. Además, en esta fase se preparó una página web en \textit{Sharepoint} en donde ir actualizando semanalmente todo el contenido del proyecto.

\subsection{Reuniones de grupo}

Para conocer en detalle la sección administrativa del centro se organizaron reuniones semanales con un grupo pequeño de trabajadores a modo de muestra que conocieran bien su puesto de trabajo.
Se realizaron un total de seis reuniones a lo largo de dos meses.
Estas se pueden separar en dos fases, por un parte las cuatro primeras reuniones se centraron en recoger información de todas las tareas administrativas que se realizan en los puestos administrativos de cara a definir un estándar y, por otras parte, las dos últimas reuniones se centraron en obtener un listado de problemas su posterior análisis de las causas raíz.

\subsubsection{Reunión 1: toma de contacto}

En esta primera reunión asistieron los dos subdirectores de gestión. Fue una presentación corta en la que se explicó al grupo de trabajo la necesidad de realizar una estandarización de las tareas administrativas de la sección del centro con el fin de analizar las cargas de trabajo reales. Además, se propuso de analizar mediante la técnica de los cinco porqués las causas raíz de los problemas administrativos.

\subsubsection{Reunión 2: citas previas}

Dado que la gestión de citas principales, y consecuentemente más tiempo consumen, fue el primer proceso que se trató. Se explicó con detalle los distintos tipos de citas: médico de familia, enfermera, trabajador social, ginecología, extracción y radiología. Además de explicar el proceso, se entregó una copia a modo de ejemplo de los volantes de extracciones y de radiología en los cuales el facultativo especifica qué pruebas concretas se deben de realizar al paciente.

\subsubsection{Reunión 3: }

\subsection{Redacción del informe}

Finalmente, tras recoger todos los datos necesarios se procede a analizar los resultados y a redactar la memoria del trabajo.

\section{Estimación de costes}


\subsection{Costes de personal}

Los costes de personal corresponden a todas las personas implicadas en el proyecto que han dedicado parte de su jornada laboral al mismo: subdirectores y administrativos principalmente.
En la Tabla~\ref{tab:horas-trabajadas} se muestra el reparto de horas trabajadas por cada grupo de personal y por fase de proyecto.

\begin{table}
    \centering
    \begin{tabular}{llll}
        \toprule
        Fases               & Ingeniero & Subdirectores & Administrativos \\
        \midrule
        Propuesta           &           &               &                 \\
        Preparación         &           &               &                 \\
        Reuniones de grupo  &           &               &                 \\
        Elaboración informe &           &               &                 \\
        \bottomrule
    \end{tabular}
    \caption{Desglose de horas dedicadas por personal y fase}
    \label{tab:horas-trabajadas}
\end{table}

Una vez obtenida la dedicación en horas se procede a calcular los costes de personal. Para ello se utilizará la tabla de retribuciones salariales de la Gerencia Regional de Salud correspondiente al año vigente, 2023, para calcular el coste por hora trabajada. Es importante, tener en cuenta que el número total de horas trabajadas por año en el Sacyl es de 1645 horas para turnos fijos diurnos.

\begin{itemize}
    \item Subdirector de Gestión: 34.40 euros/hora
    \item Administrativo: 11.78 euros/hora
    \item Ingeniero: 23.13 euros/hora
\end{itemize}

Finalmente, en la Tabla~\ref{tab:coste-horas} se muestran los costes totales de personal para todo el proyecto.

\begin{table}
    \centering
    \begin{tabular}{llll}
        \toprule
        Fases               & Ingeniero & Subdirectores & Administrativos \\
        \midrule
        Propuesta           &           &               &                 \\
        Preparación         &           &               &                 \\
        Reuniones de grupo  &           &               &                 \\
        Elaboración informe &           &               &                 \\
        \bottomrule
    \end{tabular}
    \caption{Coste de personal por fase y posición}
    \label{tab:coste-horas}
\end{table}

\subsection{Costes de amortización}

En este apartado se han calculado los costes de las amortizaciones de los equipos utilizados en el proyecto.
Fue necesario el uso de un ordenador junto con un conjunto de periféricos.
También se incluyen las licencias de los distintos programas informáticos utilizados.

\begin{table}
    \centering
    \begin{tabular}{ll}
        \toprule
        Equipo              & Coste \\
        \midrule
        Ordenador           &       \\
        Licencia Windows 10 &       \\
        Licencia Office 365 &       \\
        Impresora Kyocera   &       \\
        \bottomrule
    \end{tabular}
    \caption{Coste de los equipos amortizados}
    \label{tab:coste-equipos}
\end{table}

\subsection{Costes indirectos}

El coste indirecto es aquel que afecta al proyecto pero que no puede imputarse a ninguna de las fases o etapas del mismo.
En la Tabla~\ref{tab:coste-indirecto} se resumen los costes indirectos del proyecto.

\begin{table}
    \centering
    \begin{tabular}{ll}
        \toprule
        Coste        & Coste \\
        \midrule
        Alquiler     &       \\
        Mobiliario   &       \\
        Electricidad &       \\
        Internet     &       \\
        Transporte   &       \\
        \midrule
        TOTAL        &       \\
        \bottomrule
    \end{tabular}
    \caption{Coste indirectos del proyecto}
    \label{tab:coste-indirecto}
\end{table}