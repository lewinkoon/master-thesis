En este último capítulo se plantean las conclusiones del proyecto y el trabajo a futuro que sería conveniente realizar en este campo.

\section{Conclusiones finales}

El presente trabajo ha tratado de aplicar técnicas de mejora continua con el fin de crear un estándar de funciones y tareas en los puestos administrativos del Centro de Salud de Laguna de Duero.

Primeramente se formó un grupo de trabajo con tres de los siete miembros del equipo: Asunción, María Ángeles y Óscar.
Con ellos se organizaron un total de seis reuniones a lo largo de dos meses en la que detallaron todas las tareas llevadas a cabo en los puestos administrativos. Además, en las dos últimas reuniones se comentaron los problemas más graves que afectaban a la sección.

Tras analizar todas la información recopilada se procedió a crear unas listas estándar de tareas estándar para cada proceso.
En ellas se especifica el tipo de puesto, quién demanda realizar la tarea y una estimación cualitativa del nivel de carga de trabajo que implica.
Además, para cada uno de los procesos principales se ha creado un diagrama de flujo siguiendo el estándar BPMN con la finalidad de hacer entrender mejor al lector.

Finalmente, se aplicó la herramienta de los cinco porqués para obtener las causas raíz de los problemas que más afectan al flujo de trabajo en la sección.
Analizando las causas de los distintos problemas se crearon distintas propuestas específicas para cada tema.

\section{Trabajo futuro}

Dado que este trabajo se centra principalmente en realizar la ``fotografía'' al escenario actual de la sección administrativa del centro de salud y en proponer mejoras y propuestas concretas, conviene que los siguientes trabajos continúen en esta misma línea. Concretamente, conviene que tras este trabajo se valore la viabilidad de las distintas propuestas y se implantes las que se crean más necesarias. Todo esto sin olvidar que con la implantación de mejoras se debe de realizar un seguimiento de la efectividad de cada una, por ejemplo en base a KPIs.