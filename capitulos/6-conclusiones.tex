En este último capítulo se plantean las conclusiones del proyecto y el trabajo a futuro que sería conveniente realizar en este campo.

\section{Conclusiones finales}

El presente trabajo ha tratado de aplicar técnicas de mejora continua con el fin de crear un estándar de funciones y tareas en los puestos administrativos del Centro de Salud de Laguna de Duero.

Primeramente se formó un grupo de trabajo con tres de los siete miembros del equipo: Asunción, María Ángeles y Óscar.

Durante dos meses se realizaron seis reuniones con el grupo de trabajo para finalmente obtener las listas de trabajo estandarizadas para los puestos de ventanilla e interior (ver Tablas~\ref{tab:tareas-interior} y \ref{tab:tareas-ventanilla}). En ellas se especifica el tipo de proceso al que pertenece la tarea y una estimación del nivel de carga de trabajo que implica la tarea. Además, para cada uno de los procesos principales se ha creado un diagrama de flujo siguiendo el estándar BPMN con la finalidad de hacer entrender mejor al lector

\section{Trabajo futuro}

Dado que este trabajo, es