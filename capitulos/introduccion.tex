\section{Contexto}

En los últimos años, tras el reciente periodo de pandemia y la situación económica del país, ha quedado patente la necesidad de una buena gestión sanitaria por parte de la administración con el fin de atender la creciente demanda de servicios asistenciales por parte de la población haciendo un uso eficiente de los recursos sanitarios.

La correcta gestión de los recursos en la sanidad es un tema que en menor o mayor grado concierne a todos los ciudadanos por un motivo principal: el derecho a la protección de la salud. Este se reconoce en el artículo 43 de la Constitución y se concreta en la Ley General de Sanidad (Ley 14/1986), que establece su financiación pública, universalidad y gratuidad; su descentralización autonómica y su integración en el Sistema Nacional de Salud (SNS). En definitiva, todas las personas tienen derecho a una atención sanitaria de calidad en condiciones de igualdad.

En base a esta premisa, recientemente se han sucedido numerosas manifestaciones en defensa de la sanidad pública por parte de distintas asociaciones y representantes que proponen un aumento de las partidas destinas a equipos y personal del sistema sanitario público. Este trabajo se plantea desde otra punto de vista

% El Hospital Universitario Río Hortega de Valladolid, como su propio nombre indica, es un hospital universitario, en el que se da la oportunidad a todos los alumnos de la Universidad de Valladolid, independientemente de la rama a la que pertenezcan, a formarse y aprender en el ámbito de la salud donde se integran no sólo las ciencias de la salud sino también aquellas ramas del conocimiento referidas a técnicas sanitarias como la informática de la salud, técnicas de provisión de recursos sanitarios u otras muchas destrezas.
% El presente proyecto se ha realizado bajo un convenio de cooperación educativa entre la Universidad de Valladolid y el Hospital Universitario del Río Hortega en su unidad de procesos, con la premisa de reforzar su carácter universitario y conseguir una organización que genere conocimiento a través de su labor investigadora, que sea abierta a la sociedad, que utilice eficientemente los recursos y que sea sostenible, responsable y respetuosa con el medio ambiente.

\section{Motivación personal}

El sector sanitario es un campo en el que no es común encontrar a un ingeniero industrial. Y cuando esto sucede, normalmente se suele asociar a aspectos técnicos como pueden ser el mantenimiento de las instalaciones térmicas y eléctricas o la ingeniería biomédica.

\section{Objetivos}

\section{Alcance}

% El área de estudio comprende una unidad de hospitalización concreta en la que el sistema de doble cajón o “Kanban” se ha implantado de forma experimental en productos farmacéuticos, para en un futuro si se cumplen los objetivos citados en el punto 1.3 de este trabajo, extender su implantación y funcionamiento al resto de unidades.

% El proyecto consta de varias partes, en primer lugar, el estudio previo del consumo en el año 2021 del botiquín farmacéutico de la unidad de hospitalización de Medicina Interna nos permitirá establecer un pacto basado en el consumo real. Para ello, se deberá comprender el funcionamiento del Sistema Kanban, ya incorporado para el material fungible y que ahora da paso a la creación de este mismo sistema para el botiquín farmacéutico.

% Una segunda parte será el planteamiento del catálogo de artículos implicados en el estudio, cantidades y ubicaciones con ayuda del personal de enfermería involucrado; este catálogo se distribuye en los armarios y habitaciones prestadas al estudio, se adecua el mobiliario en términos de una mayor accesibilidad, se etiquetan todos los artículos e incluyen en el programa informático del HURH y la supervisora de la unidad se cerciora de que el personal de enfermería implicado conoce el funcionamiento de este nuevo sistema.

% Por último, su puesta en marcha, el personal oportuno deberá asegurarse de que las etiquetas se encuentran debidamente colocadas en el panel antes de primera hora de la mañana, que es cuando se hará la lectura de los pedidos y el departamento de logística empezará a tramitarlo. El mismo día que se comience, se iniciará un estudio sobre los consumos de las dos unidades con los catálogos planteados.

\section{Estructura memoria}

Esta memoria sigue la estructura especificada en la guía docente para la asignatura TFM del Máster en Ingeniería Industrial de la Escuela de Ingenieros Industriales (Universidad de Valladolid). \cite{lion2010}
