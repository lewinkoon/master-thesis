\section{Contexto}

En los últimos años, tras el reciente periodo de pandemia y la situación económica del país, ha quedado patente la necesidad de una buena gestión sanitaria por parte de la administración con el fin de atender la creciente demanda de servicios asistenciales por parte de la población haciendo un uso eficiente de los recursos sanitarios.

La correcta gestión de los recursos en la sanidad es un tema que en menor o mayor grado concierne a todos los ciudadanos por un motivo principal: el derecho a la protección de la salud. Este se reconoce en el artículo 43 de la Constitución y se concreta en la Ley General de Sanidad (Ley 14/1986), que establece su financiación pública, universalidad y gratuidad; su descentralización autonómica y su integración en el Sistema Nacional de Salud (SNS). En definitiva, todas las personas tienen derecho a una atención sanitaria de calidad en condiciones de igualdad.

En base a esta premisa, recientemente se han sucedido numerosas manifestaciones en defensa de la sanidad pública por parte de distintas asociaciones y representantes que proponen un aumento de las partidas destinadas a equipos y personal del sistema sanitario público. Este trabajo va un paso más allá, presentando distinas propuestas en base a técnicas de mejora continua que mejoren la eficiencia de los recursos ya disponibles, tanto económicos como de personal.

\section{Motivación personal}

El sector sanitario es un campo en el que no es común encontrar a un ingeniero industrial. Y cuando esto sucede, normalmente se suele asociar a aspectos técnicos como pueden ser el mantenimiento de las instalaciones térmicas y eléctricas o la ingeniería biomédica. Más extraño es, encontrar a un ingeniero involucrado en la mejora de procesos asistenciales.

Fue lo exótico de este campo junto con la posibilidad de poder dedicar mi tiempo a ayudar, aunque sea de forma indirecta, a los demás en un sector tan social como es el sanitario lo que me llevó a realizar este trabajo fin de máster.

La ingeniería de procesos, al menos tal y como se imparte en la carrera, está orientada principalmente a la industria manufacturera en cuanto a producción de bienes de equipo. Todas las metodologías aprendidas, kaizen, 5s, JIT o kanban son muchas veces aplicables a los procesos asistenciales sustituyendo el producto material por personas.

\section{Objetivos}

El objetivo principal de este trabajo fin de máster es analizar la problemática y plantear propuestas de mejora en el centro de salud de Laguna de Duero.

Para conseguir este objetivo se plantean los siguientes subobjetivos.
\begin{enumerate}
    \item Crear un grupo de trabajo con los representantes del personal administrativo del centro.
    \item Realizar una lista de tareas comunes al personal administrativo tanto de ventanilla como de interior
    \item Elaborar una tabla con los problemas de mayor importancia y sus respectivas causas
    \item Plantear mejoras en los distintos procesos que se llevan a cabo
\end{enumerate}

\section{Alcance}

El área de estudio comprende la sección administrativa del centro de salud de Laguna de Duero (Valladolid) perteneciente al Área de Salud Valladolid Oeste. La decisión de elegir este centro fue motivada por la gran variedad de servicios que ofrece: extracciones, salud bucodental, fisioterapia, radioogía y un punto de atención continuada. Además, el centro se encuentra a medio camino entre la tipología rural y urbana haciendolo idóneo para realizar este proyecto en vista de aplicarlo al resto de centros de salud del área.

Dado que este proyecto se ha propuesto desde la Dirección de Gestión de la Gerencia de Atención Primaria, solamente se ha involucrado al personal administrativo. Si se cumplen los objetivos, en el futuro se pretende incluir en el grupo de trabajo a representantes de los demás grupos de interés involucrados en los procesos del centro de salud: médicos, enfermeras, auxiliares de enfermería y celadores entre otros.

\section{Estructura memoria}

Esta memoria sigue la estructura especificada en la guía docente para la asignatura TFM del Máster en Ingeniería Industrial de la Escuela de Ingenieros Industriales (Universidad de Valladolid).

Siguiendo al presente capítulo de introducción, donde se expone el contexto, la motivación, el alcance y los objetivos de este trabajo, se encuentran los siguientes capítulos:

\begin{itemize}
    \item Capítulo 2. El sistema sanitario en Castilla y León
    \item Capítulo 3. Métodología de la mejora continua en servicios
    \item Capítulo 4. Desarrollo
    \item Capítulo 5. Estudio económico
    \item Capítulo 6. Conclusiones
\end{itemize}

Finalmente se adjunta las referencias bibliográficas utilizadas.
