\section{Contexto}

El Hospital Universitario Río Hortega de Valladolid (en adelante HURH), como su propio nombre indica, es un hospital universitario, en el que se da la oportunidad a todos los alumnos de la Universidad de Valladolid, independientemente de la rama a la que pertenezcan, a formarse y aprender en el ámbito de la salud donde se integran no sólo las ciencias de la salud sino también aquellas ramas del conocimiento referidas a técnicas sanitarias como la informática de la salud, técnicas de provisión de recursos sanitarios u otras muchas destrezas. El presente proyecto se ha realizado bajo un convenio de cooperación educativa entre la Universidad de Valladolid y el Hospital Universitario del Río Hortega en su unidad de procesos, con la premisa de reforzar su carácter universitario y conseguir una organización que genere conocimiento a través de su labor investigadora, que sea abierta a la sociedad, que utilice eficientemente los recursos y que sea sostenible, responsable y respetuosa con el medio ambiente.

\section{Motivación personal}

Antes de empezar a estudiar el Máster en Ingeniería Industrial en la Universidad de Valladolid, ya sentía cierta curiosidad por la mecánica del sector sanitario, además, en mi familia la sanidad siempre ha estado muy presente como profesión en todas las generaciones; es por eso por lo que, desde mi afán por la logística, la cual es mi rama de economía favorita, la he tratado de fusionar con la sanidad. Con la pandemia provocada por el Coronavirus SARS-CoV-2 se ha multiplicado el estímulo que hemos recibido cada día acerca de la importancia de un correcto funcionamiento del sistema hospitalario y su afán de mejorar y, es cuando he comprendido que cada proceso que se da en un hospital repercute en otro y sobre todo en los pacientes, al igual que en una empresa, o en una cadena de montaje. El HURH me ha brindado una gran oportunidad de llevar a cabo este proyecto, el cual, es uno de los hospitales pioneros en España y de referencia en Castilla y León por su constante actualización tecnológica y desarrollo de las tecnologías de la
comunicación.

\section{Objetivos}

\section{Alcance}

El área de estudio comprende una unidad de hospitalización concreta en la que el sistema de doble cajón o “Kanban” se ha implantado de forma experimental en productos farmacéuticos, para en un futuro si se cumplen los objetivos citados en el punto 1.3 de este trabajo, extender su implantación y funcionamiento al resto de unidades.

El proyecto consta de varias partes, en primer lugar, el estudio previo del consumo en el año 2021 del botiquín farmacéutico de la unidad de hospitalización de Medicina Interna nos permitirá establecer un pacto basado en el consumo real. Para ello, se deberá comprender el funcionamiento del Sistema Kanban, ya incorporado para el material fungible y que ahora da paso a la creación de este mismo sistema para el botiquín farmacéutico.

Una segunda parte será el planteamiento del catálogo de artículos implicados en el estudio, cantidades y ubicaciones con ayuda del personal de enfermería involucrado; este catálogo se distribuye en los armarios y habitaciones prestadas al estudio, se adecua el mobiliario en términos de una mayor accesibilidad, se etiquetan todos los artículos e incluyen en el programa informático del HURH y la supervisora de la unidad se cerciora de que el personal de enfermería implicado conoce el funcionamiento de este nuevo sistema.

Por último, su puesta en marcha, el personal oportuno deberá asegurarse de que las etiquetas se encuentran debidamente colocadas en el panel antes de primera hora de la mañana, que es cuando se hará la lectura de los pedidos y el departamento de logística empezará a tramitarlo. El mismo día que se comience, se iniciará un estudio sobre los consumos de las dos unidades con los catálogos planteados.

\section{Estructura memoria}

Esta memoria sigue la estructura especificada en la guía docente para la asignatura TFM del Máster en Ingeniería Industrial de la Escuela de Ingenieros Industriales (Universidad de Valladolid). \cite{lion2010}
