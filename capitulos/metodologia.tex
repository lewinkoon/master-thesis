En el siguiente capítulo se desarrolla la metodología utilizada en el proyecto.

\section{Fundamentos de Lean Management}

La gestión Lean es un concepto moderno para la optimización de procesos en toda la cadena de valor.
Se centra en hacer que las ineficiencias, o desperdicios, sean transparentes y en alterarlas para convertirlas en actividades que añadan valor.
La cadena de valor abarca en este contexto desde los proveedores, pasando por las propias operaciones, hasta laos clientes.
Las ineficiencias son todo aquello, por ejemplo, una actividad, un proceso y un producto, que se considera algo por lo que los clientes no están dispuestos a pagar o a gastar medios financieros.
El cliente es el punto central del concepto lean.
Los principales objetivos de la filosofía lean son crear valor para el cliente mediante la optimización de los recursos y crear un flujo de trabajo constante basado en las demandas reales de los clientes. Busca eliminar cualquier pérdida de tiempo, esfuerzo o dinero identificando cada paso en un proceso de negocio y luego revisando o recortando los pasos que no crean valor.
La filosofía tiene sus raíces en Japón y las operaciones, pero actualmente está ampliamente extendida por todo el mundo y las industrias. Sus puntos centrales son los siguientes:

\begin{itemize}
    \item Poner al cliente en el centro de la operación.
    \item Definir el valor y el valor añadido desde el punto de vista del cliente final.
    \item Eliminar todos los residuos en todos los ámbitos de la cadena de valor.
    \item Mejorar continuamente todas las actividades, procesos, objetivos y personas.
    \item Situar a las personas en el centro de los servicios y procesos de valor añadido.
\end{itemize}

La gestión lean facilita el liderazgo y la responsabilidad compartido y la mejora continua garantiza que cada empleado contribuya al proceso de mejora.
El método de gestión actúa como guía para construir una organización sólida y de éxito que progresa constantemente, identificando los problemas reales y resolviéndolos.
Lean se basa en el sistema de producción Toyota, creado a finales de los años cuarenta.
Toyota puso en práctica los cinco principios de la gestión ajustada con el objetivo de reducir la cantidad de procesos que no producían valor, lo que se conoce como Toyota Way.
Al aplicar los cinco principios, encontraron mejoras significativas en eficiencia, productividad, rentabilidad y duración de los ciclos.
Lean incorpora cinco principios rectores que son utilizados por los directivos de una organización como directrices de la metodología. Los cinco principios son:

\begin{itemize}
    \item Identificar el valor en todos los procesos de la cadena de valor.
    \item Realización de mapas de flujo de valor.
    \item Crear un flujo de trabajo continuo.
    \item Establezca un sistema pull en el que los clientes sean el centro de atención.
\end{itemize}

Identificar el valor, el primer paso en la gestión lean, significa encontrar el problema que el cliente necesita resolver y convertir el producto en la solución.
En concreto, el producto debe ser la parte de la solución por la que el cliente esté dispuesto a pagar.
Cualquier proceso o actividad que no añada valor, es decir, que no aporte utilidad y el cliente no está dispuesto a pagar por ello, importancia o valor al producto fnal se considera residuo y debe eliminarse.

El mapeo de la cadena de valor se refiere al proceso de mapear el flujo de trabajo de la empresa, incluyendo todas las acciones y personas que contribuyen al proceso de creación y entrega del producto final al consumidor.
El mapeo del flujo de valor ayuda a los directivos a visualizar qué procesos están dirigidos por qué equipos y a identificar a las personas responsables de medir, evaluar y mejorar el proceso.
Esta visualización ayuda a los directivos a determinar qué partes del sistema no aportan valor al flujo de trabajo.

Crear un flujo de trabajo continuo significa garantizar que el flujo de trabajo de cada equipo progrese sin problemas y evitar las interrupciones o cuellos de botella que pueden producirse con el trabajo en equipos interfuncionales.
Kanban, una técnica de gestión ajustada que utiliza una señal visual para desencadenar la acción, se utiliza para facilitar la comunicación entre equipos para que puedan abordar lo que hay que hacer y cuándo hay que hacerlo.
En el proceso de trabajo total en una colección de partes más pequeñas y visualizar el flujo de trabajo en este sentido facilitan la eliminación factible de interrupciones y interrupciones y bloqueos.

Desarrollar un sistema \textit{pull} asegura que el flujo de trabajo continuo se mantenga estable y garantiza que los equipos entreguen las asignaciones de trabajo más rápido y con menos esfuerzo. Un sistema pull es una técnica lean específica que reduce los residuos de cualquier producción. Garantiza que sólo se inicie un nuevo trabajo si hay demanda del mismo, lo que ofrece la ventaja de minimizar los gastos generales y optimizar los costes de almacenamiento.

El último principio es la mejora continua y puede considerarse el paso más importante del método de gestión ajustada.
Facilitar la mejora continua se refiere a una serie de técnicas que se utilizan para identificar lo que una organización ha hecho, lo que necesita hacer, los posibles obstáculos que puedan surgir y cómo todos los miembros de la organización pueden mejorar sus procesos de trabajo.
El sistema lean no está aislado ni es inmutable, por lo que pueden surgir problemas en cualquiera de los otros cuatro pasos.
Asegurarse de que todos los empleados contribuyen a la mejora continua del flujo de trabajo protege a la organización cuando surgen problemas.
La dirección debe crear un entorno y una cultura en los que todos los empleados puedan trabajar de acuerdo con los cinco principios.