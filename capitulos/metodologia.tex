En el siguiente capítulo se desarrolla la metodología utilizada en el proyecto.

\section{Fundamentos de Lean Management}

La gestión Lean es un concepto moderno para la optimización de procesos en toda la cadena de valor.
Se centra en hacer que las ineficiencias, o desperdicios, sean transparentes y en alterarlas para convertirlas en actividades que añadan valor.
La cadena de valor abarca en este contexto desde los proveedores, pasando por las propias operaciones, hasta laos clientes.
Las ineficiencias son todo aquello, por ejemplo, una actividad, un proceso y un producto, que se considera algo por lo que los clientes no están dispuestos a pagar o a gastar medios financieros.

El cliente es el punto central del concepto lean.
Los principales objetivos de la filosofía lean son crear valor para el cliente mediante la optimización de los recursos y crear un flujo de trabajo constante basado en las demandas reales de los clientes. Busca eliminar cualquier pérdida de tiempo, esfuerzo o dinero identificando cada paso en un proceso de negocio y luego revisando o recortando los pasos que no crean valor.

La filosofía tiene sus raíces en Japón y las operaciones, pero actualmente está ampliamente extendida por todo el mundo y las industrias. Sus puntos centrales son los siguientes:

\begin{itemize}
    \item Poner al cliente en el centro de la operación.
    \item Definir el valor y el valor añadido desde el punto de vista del cliente final.
    \item Eliminar todos los residuos en todos los ámbitos de la cadena de valor.
    \item Mejorar continuamente todas las actividades, procesos, objetivos y personas.
    \item Situar a las personas en el centro de los servicios y procesos de valor añadido.
\end{itemize}

La gestión lean facilita el liderazgo y la responsabilidad compartido y la mejora continua garantiza que cada empleado contribuya al proceso de mejora.
El método de gestión actúa como guía para construir una organización sólida y de éxito que progresa constantemente, identificando los problemas reales y resolviéndolos.
Lean se basa en el sistema de producción Toyota, creado a finales de los años cuarenta.
Toyota puso en práctica los cinco principios de la gestión ajustada con el objetivo de reducir la cantidad de procesos que no producían valor, lo que se conoce como Toyota Way.

Al aplicar los cinco principios, encontraron mejoras significativas en eficiencia, productividad, rentabilidad y duración de los ciclos.
Lean incorpora cinco principios rectores que son utilizados por los directivos de una organización como directrices de la metodología. Los cinco principios son:

\begin{itemize}
    \item Identificar el valor en todos los procesos de la cadena de valor.
    \item Realización de mapas de flujo de valor.
    \item Crear un flujo de trabajo continuo.
    \item Establezca un sistema pull en el que los clientes sean el centro de atención.
\end{itemize}

Identificar el valor, el primer paso en la gestión lean, significa encontrar el problema que el cliente necesita resolver y convertir el producto en la solución.
En concreto, el producto debe ser la parte de la solución por la que el cliente esté dispuesto a pagar.
Cualquier proceso o actividad que no añada valor, es decir, que no aporte utilidad y el cliente no está dispuesto a pagar por ello, importancia o valor al producto fnal se considera residuo y debe eliminarse.

El mapeo de la cadena de valor se refiere al proceso de mapear el flujo de trabajo de la empresa, incluyendo todas las acciones y personas que contribuyen al proceso de creación y entrega del producto final al consumidor.
El mapeo del flujo de valor ayuda a los directivos a visualizar qué procesos están dirigidos por qué equipos y a identificar a las personas responsables de medir, evaluar y mejorar el proceso.
Esta visualización ayuda a los directivos a determinar qué partes del sistema no aportan valor al flujo de trabajo.

Crear un flujo de trabajo continuo significa garantizar que el flujo de trabajo de cada equipo progrese sin problemas y evitar las interrupciones o cuellos de botella que pueden producirse con el trabajo en equipos interfuncionales.
Kanban, una técnica de gestión ajustada que utiliza una señal visual para desencadenar la acción, se utiliza para facilitar la comunicación entre equipos para que puedan abordar lo que hay que hacer y cuándo hay que hacerlo.
En el proceso de trabajo total en una colección de partes más pequeñas y visualizar el flujo de trabajo en este sentido facilitan la eliminación factible de interrupciones y interrupciones y bloqueos.

Desarrollar un sistema \textit{pull} asegura que el flujo de trabajo continuo se mantenga estable y garantiza que los equipos entreguen las asignaciones de trabajo más rápido y con menos esfuerzo. Un sistema pull es una técnica lean específica que reduce los residuos de cualquier producción. Garantiza que sólo se inicie un nuevo trabajo si hay demanda del mismo, lo que ofrece la ventaja de minimizar los gastos generales y optimizar los costes de almacenamiento.

El último principio es la mejora continua y puede considerarse el paso más importante del método de gestión ajustada.
Facilitar la mejora continua se refiere a una serie de técnicas que se utilizan para identificar lo que una organización ha hecho, lo que necesita hacer, los posibles obstáculos que puedan surgir y cómo todos los miembros de la organización pueden mejorar sus procesos de trabajo.
El sistema lean no está aislado ni es inmutable, por lo que pueden surgir problemas en cualquiera de los otros cuatro pasos.
Asegurarse de que todos los empleados contribuyen a la mejora continua del flujo de trabajo protege a la organización cuando surgen problemas.
La dirección debe crear un entorno y una cultura en los que todos los empleados puedan trabajar de acuerdo con los cinco principios.

\section{Origen de la filosofía lean}

\subsection{Primeros avances de la gestión Lean}

Los primeros desarrollos de las herramientas lean se remontan a los primeros tiempos de la industrialización.
Con el aumento de la demanda de los clientes, los empresarios intentaban implantar procesos que aceleraran y aumentaran la producción.
Eli Whitney es más famoso como inventor de la desmotadora de algodón.
Sin embargo, la desmotadora fue un logro menor comparado con su perfección de las piezas intercambiables.
Whitney lo desarrolló hacia 1799, cuando aceptó un contrato del ejército estadounidense para la fabricación de 10.000 mosquetes al precio increíblemente bajo de 13,40 dólares por cada arma.

Durante los 100 años siguientes, los fabricantes se ocuparon principalmente de tecnologías individuales.
Durante este tiempo se desarrolló nuestro sistema de planos de ingeniería, se perfeccionaron las máquinas-herramienta modernas y los procesos a gran escala, como el proceso Bessemer para fabricar acero, ocuparon el centro de atención.
Mientras los productos pasaban de un proceso discreto al siguiente a través del sistema logístico y dentro de las fábricas, poca gente se preocupaba por:

\begin{itemize}
    \item Qué ocurre entre un proceso y otro.
    \item Cómo se organizaron los múltiples procesos dentro de la fábrica
    \item Cómo funcionaba la cadena de procesos como sistema
    \item Cómo realizó una tarea cada trabajador
\end{itemize}

\subsection{Ford y el Taylorismo}

Esto cambió a finales de la década de 1890 con el trabajo de los primeros ingenieros industriales.
Frederick W. Taylor empezó a estudiar a cada trabajador y sus métodos de trabajo.
El resultado fueron los estudios sobre la gestión del tiempo, el tiempo por ciclo y las operaciones de trabajo estandarizadas.
Llamó a sus ideas gestión científica.
Taylor fue un directivo y una personalidad controvertidos.
El concepto de aplicar la ciencia a la gestión era sólido, pero Taylor simplemente ignoró las ciencias del comportamiento.
Además, tenía una actitud peculiar hacia los trabajadores de las fábricas.
Frank Gilbreth añadió el estudio del movimiento e inventó los gráficos de procesos.
Los gráficos de procesos centraban la atención en todos los elementos del trabajo, incluidos los elementos sin valor añadido que normalmente se producen entre los elementos "oficiales".

Lillian Gilbreth introdujo la psicología al estudiar las motivaciones de los trabajadores y cómo las actitudes afectaban al resultado de un proceso.
Hubo, por supuesto, muchos otros colaboradores.
Estas personas fueron las que originaron la idea de "eliminar los residuos", un principio clave del JIT y la fabricación ajustada.
Aunque existen ejemplos de procesos de fabricación rigurosos que se remontan al Arsenal de Venecia en la década de 1450, la primera persona que integró realmente todo un proceso de producción fue Henry Ford.
En 1913, en Highland Park (Michigan), unió las piezas intercambiables con el trabajo estándar y el transporte en movimiento para crear lo que denominó "producción en cadena"
El público lo entendió en la dramática forma de la cadena de montaje en movimiento, pero desde el punto de vista del ingeniero de fabricación, los avances en realidad iban mucho más allá.

Ford alineaba los pasos de fabricación en la secuencia del proceso siempre que era posible, utilizando máquinas especiales y calibradores "go/no-go" para fabricar y ensamblar los componentes del vehículo en pocos minutos y entregar los componentes perfectamente ajustados directamente a la línea de producción.
Se trataba de una ruptura verdaderamente revolucionaria con respecto a las prácticas de taller del sistema americano, que consistía en máquinas de uso general agrupadas por procesos, que fabricaban piezas que acababan convirtiéndose en productos acabados tras un buen rato de retoques (ftting) en el subensamblaje y el ensamblaje final.
El problema del sistema de Ford no era el flujo: era capaz de dar la vuelta a los inventarios de toda la empresa cada pocos días.
Más bien era su incapacidad para ofrecer variedad.
El Modelo T no sólo se limitaba a un color, que era negro.

También se limitó a una especificación, de modo que todos los chasis del Modelo T eran esencialmente idénticos hasta el final de la producción en 1926.
El cliente sí podía elegir entre cuatro o cinco estilos de carrocería, una característica añadida por proveedores externos al final de la línea de producción.
De hecho, parece que prácticamente todas las máquinas de la Ford Motor Company trabajaban con un único número de pieza, y prácticamente no había cambios.
Cuando el mundo quiso variedad, incluyendo ciclos de modelos más cortos que los 19 años del Modelo T, Ford pareció perder el rumbo.

Otros fabricantes de automóviles respondieron a la necesidad de muchos modelos, cada uno con muchas opciones, pero con sistemas de producción cuyos pasos de diseño y fabricación retrocedían hacia áreas de proceso con tiempos de producción mucho más largos.
Con el tiempo, poblaron sus talleres de fabricación con máquinas cada vez más grandes que funcionaban cada vez más rápido, lo que aparentemente reducía los costes por paso del proceso, pero aumentaba continuamente los tiempos de producción y los inventarios, excepto en algunos casos (como las líneas de mecanizado de motores) en los que todos los pasos del proceso podían conectarse y automatizarse.
Peor aún, los desfases entre las fases del proceso y las complejas rutas de las piezas exigían sistemas de gestión de la información cada vez más sofisticados, que culminaron en los sistemas informatizados de planificación de necesidades de materiales (MRP).

\subsection{Sistema de Producción Toyota (TPS)}

Cuando Kiichiro Toyoda, Taiichi Ohno y otros miembros de Toyota analizaron esta situación en década de 1930, y más intensamente justo después de la II Guerra Mundial, se les ocurrió que una serie de una serie de sencillas innovaciones podría hacer más posible la continuidad continuidad en el flujo de procesos y una amplia variedad en la oferta de productos.
Así pues, retomaron Ford e inventaron el sistema de producción Toyota.

En esencia, este sistema desplazó la atención del ingeniero de fabricación de las máquinas individuales y su utilización al flujo del producto a través del proceso total.
Toyota llegó a la conclusión de que si se ajustaba el tamaño de las máquinas al volumen real necesario, se introducían máquinas con autocontrol para garantizar la calidad, se alineaban las máquinas en la secuencia del proceso, se iniciaban configuraciones rápidas para que cada máquina pudiera fabricar pequeños volúmenes de muchos números de piezas y se hacía que cada paso del proceso notificara al paso anterior sus necesidades actuales de materiales, sería posible obtener bajos costes, alta variedad, alta calidad y tiempos de producción muy rápidos para responder a los deseos cambiantes de los clientes.
El concepto del TPS se basa en un paradigma de mejora permanente y continua, la filosofía kaizen.
Además, la gestión de la información podría ser mucho más sencilla y precisa. En el libro The Machine That Changed the World (La máquina que cambió el mundo, 1990) de Womack, Jones y Roos, se describe detalladamente el proceso de pensamiento del lean. En él los autores describen que los principios del lean se basan en cinco elementos:

\begin{itemize}
    \item Especificar el valor deseado por el cliente.
    \item Identificar el flujo de valor de cada producto que aporta ese valor y cuestionar todos de los pasos desperdiciados necesarios actualmente.
    \item Hacer que el producto fluya continuamente a través de las etapas de valor añadido.
    \item Introducir tirones entre todos los pasos en los que sea posible un flujo continuo.
    \item Perfeccionar el proceso para que el número de pasos y la cantidad de tiempo e información necesarios para atender al cliente disminuyan continuamente.
\end{itemize}

\subsection{La gestión Lean en la actualidad}

Este éxito continuado ha generado en las dos últimas décadas una enorme demanda de conocimientos sobre el pensamiento lean.
Hay literalmente cientos de libros y documentos, por no hablar de los miles de artículos de los medios de comunicación que exploran el tema, y otros muchos recursos a disposición de este público cada vez más numeroso.
A medida que el pensamiento lean sigue extendiéndose por todos los países del mundo, los líderes también están adaptando las herramientas y los principios más allá de la fabricación, a la logística y la distribución, los servicios, el comercio minorista, la sanidad, la construcción, el mantenimiento e incluso la administración pública.
De hecho, la conciencia y los métodos lean están empezando a arraigar entre los altos directivos y líderes de todos los sectores en la actualidad.
Las redes de cadenas de valor en los tiempos actuales son estructuras complejas e internacionales de oferta y demanda.
Especialmente, los fabricantes japoneses muestran cómo los proveedores se integran de forma sostenible en la propia cadena de valor y actividades.
Las redes japonesas se describen como redes \textit{keiretsu}, en las que proveedores y clientes son sistemas integrados a lo largo de la cadena de valor.
En el futuro, la competitividad se decidirá en función de quién tenga la red de valor más flexible y eficiente, que incluya flujos de valor desde los proveedores de materias primas, pasando por las operaciones propias, hasta la distribución a los clientes.

\section{La gestión lean en el sector sanitario}

Lean parte del rechazo al despilfarro. Atribuido a Taiichi Ohno, el sistema Lean se desarrolló en los años 50 y 60 para ofrecer la mejor calidad, el menor coste y el menor plazo de entrega mediante la eliminación de los residuos. El término japonés para lo que las empresas estadounidenses suelen clasificar como despilfarro es \textit{muda} y fue definido por Fujio Cho de Toyota como "cualquier cosa que no sea la cantidad mínima de equipo, espacio y tiempo del trabajador, que son absolutamente esenciales para añadir valor al producto". La presencia de este tipo de residuos en un sistema repercute negativamente en el plazo de entrega, el coste y la calidad. A principios de los años 80, empresas de otros sectores, como el sanitario, comprendieron que la introducción de los principios lean conllevaría varias ventajas. El despilfarro en la sanidad puede describirse, entre otros elementos, en el transporte excesivo de medicamentos o pacientes, el tiempo de espera para los tratamientos o la infrautilización de equipos y máquinas en los hospitales. Además, las duplicaciones e ineficiencias del personal de enfermería también pueden repercutir en la creación de residuos.

Aunque la metodología de mejora empresarial lean se desarrolló inicialmente para mejorar la calidad y la productividad de las fábricas de automóviles, se ha utilizado con gran éxito en industrias y entornos de todo tipo, como el desarrollo de software, la administración pública, el comercio minorista y otros entornos de servicios. Las organizaciones sanitarias, en particular, han descubierto que el enfoque puede utilizarse para reducir costes y mejorar la calidad y la satisfacción de los pacientes al mismo tiempo. Uno de los principios básicos del lean es la eliminación del despilfarro, que se define como todo aquello que no añade valor al cliente. Los profesionales se centran en ocho tipos específicos de residuos (originalmente eran siete, pero hablaremos de ellos más adelante). Son tan comunes en la sanidad como en la industria. Por lo tanto, la gestión ajustada tiene como objetivo eliminar, por ejemplo, los tiempos de espera o el exceso de medicación.

\subsection{Transporte}

El despilfarro del transporte se produce cuando los materiales se trasladan de un lugar a otro de forma ineficaz. En sanidad se produce cuando:

\begin{itemize}
    \item Los pacientes son trasladados de un departamento a otro o de una habitación a otra.
    \item Los medicamentos se trasladan de la farmacia al lugar donde se necesitan
    \item Los suministros se trasladan del almacén a la planta
\end{itemize}

Algunos de estos transportes se consideran residuos "necesarios" que hay que minimizar aunque no puedan eliminarse por completo.

\subsection{Inventario}

Los fabricantes han adoptado en gran medida un enfoque de inventario justo a tiempo para reducir los costes relacionados con el almacenamiento, el movimiento, el deterioro y el desperdicio. Las organizaciones sanitarias buscan hacer lo mismo en lo que se refiere a:

\begin{itemize}
    \item Medicación que esté cerca de la fecha de caducidad
    \item Exceso de consumibles
    \item Formularios preimpresos
    \item Exceso de material de cabecera
\end{itemize}

\subsection{Movimiento}

El movimiento se refiere al desplazamiento innecesario de personas dentro de una instalación. Este ocurre cuando:

\begin{itemize}
    \item La distribución de las oficinas o del hospital no es coherente con el flujo de trabajo.
    \item Los suministros no se almacenan donde se necesitan.
    \item Los equipos no están bien situados
\end{itemize}

El primer paso para combatir los despilfarros del lean es reconocerlos dentro de su organización. En la mayoría de los casos, el examen de cada uno de estos factores específicos que contribuyen con frecuencia al despilfarro conduce al descubrimiento de múltiples oportunidades de mejora. También podemos esforzarnos por eliminar el despilfarro (incluidos los clics) en los sistemas de software.

\subsection{Esperas}

En la fabricación, la espera se produce cuando las piezas no pueden salir o cuando los miembros del equipo no pueden realizar sus tareas debido a problemas, como la falta de existencias o fallos en los equipos. La espera en la atención sanitaria es un problema tanto para los pacientes como para los proveedores.

\begin{itemize}
    \item Patients in waiting rooms (or exam rooms).
    \item Funcionarios con cargas de trabajo desiguales a la espera de su próxima tarea.
    \item Pacientes de urgencias y médicos a la espera de los resultados de las pruebas.
    \item Pacientes de urgencias en espera de ingreso hospitalario
    \item Pacientes en espera de alta médica
\end{itemize}

\subsection{Sobreutilización}

En la industria manufacturera, la sobreproducción se traduce en un exceso de "trabajo en curso" o de existencias de "productos acabados" sin vender. En sanidad es más difícil de detectar, pero se produce cuando los proveedores hacen más de lo que necesita el cliente en ese momento. Incluye:

\begin{itemize}
    \item Pruebas diagnósticas innecesarias
    \item Comidas no consumidas
    \item Pedir medicamentos que el paciente no necesita
    \item Personal en horas no punta
\end{itemize}