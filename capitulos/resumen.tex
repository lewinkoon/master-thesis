\chapter*{Resumen}
GIS es una potente herramienta que puede utilizarse para localizar manantiales originados en ofiolitas. Las características únicas asociadas a estos manantiales incluyen un entorno subsuperficial reductor que reacciona a bajas temperaturas produciendo fluidos de formación de alto pH, ricos en Ca y con un alto contenido en hidrógeno y metano disueltos. Debido a sus características químicas únicas, estas zonas suelen estar asociadas a microbios y se cree que son similares a las características que permitieron la evolución de la vida en la Tierra. La localización y muestreo de estos manantiales podría ofrecer una visión más profunda de la biosfera profunda de la Tierra y de la historia de la vida en la Tierra. Tradicionalmente, los manantiales se han localizado mediante técnicas de campo costosas y lentas. El trabajo de campo puede ser peligroso. El objetivo de este estudio era desarrollar un modelo que permitiera localizar estas características geológicas únicas sin tener que ir primero al campo, ahorrando así tiempo, dinero y reduciendo los riesgos asociados a las localidades remotas sobre el terreno. Un análisis de idoneidad de yacimientos SIG funciona superponiendo datos georreferenciados existentes en un programa informático y sumando los distintos conjuntos de datos tras asignar un valor numérico a los campos importantes. Para este proyecto, he utilizado mapas de aguas superficiales y subterráneas, mapas geológicos, un mapa de suelos y un mapa de fallas de cuatro condados del norte de California. El modelo ha demostrado que es posible utilizar este tiempo de modelo y aplicarlo a un área geológica compleja para producir un mapa de campo utilizable para futuros trabajos de campo.

\chapter*{Abstract}
GIS is a powerful tool that can be used to locate springs sourced in ophiolites. The unique features associated with these springs include a reducing subsurface environment reacting at low temperatures producing high pH, Ca-rich formation fluids with high dissolved hydrogen and methane. Because of their unique chemical characteristics, these areas are often associated with microbes and are thought to be similar to the features that enabled life to evolve on Earth. Locating and sampling these springs could offer a deeper look into Earth's deep biosphere and the history of life on Earth. Springs have tradiitionally been located using expensive and time consuming field techniques. Field work can be dangerous. The goal of this study was to develop a model that could locate these unique geological features without first going into the field, thus saving time, money and reducing the risks associated with remote field localities. A GIS site suitability analysis works by overlaying existing geo-referenced data into a computer program and adding the different data sets after assigning a numerical value to the important fields. For this project, I used surface and ground water maps, geologic maps, a soil map, and a fault map for four counties in Northern California. The model has demonstrated that it is possible to use this time of model and apply it to a complex geologic area to produce a usable field map for future field work.