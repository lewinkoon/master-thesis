Una Zona Básica de Salud está formada por un conjunto de profesionales sanitarios y no sanitarios, que constituyen el equipo de Atención Primaria, y que son los responsables de la prestación de la atención de la salud a la población en su demarcación sanitaria, y ello, de forma coordinada, integral, permanente y continuada, y orientada al individuo, a la comunidad y al medio ambiente.

Entre los integrantes del equipo se encuentran médicos de familia, pediatras, enfermeros, matronas, auxiliares de enfermería, trabajadores sociales, auxiliares administrativos y celadores. Además, integrando funcionalmente, existen veterinarios y farmacéuticos.

Cada zona de salud dispone de un centro de salud, estructura dotada de los medios necesarios para la prestación de las funciones que debe desempeñar el equipo de atención primaria. Además, del centro de salud, existen consultorios locales destinados a aquellas localidades de más de 50 habitantes, donde los profesionales sanitarios atienden la demanda asistencial bajo el criterio de una correcta accesibilidad de los servicios a la población.

El equipo de atención primaria se organiza jerárquicamente bajo la supervisión del coordinador del centro de salud, nombrado de entre los facultativos del equipo, responsable de la gestión de los recursos humanos y materiales. Entre sus funciones se encuentran las siguientes:

\begin{itemize}
    \item Asumir la representación oficial del equipo
    \item Ejercer la dirección y coordinación de todo el personal en lo referente al régimen interno de funcionamiento, así como resolver los conflictos en lo refenrente a dicho régimen interno de funcionamiento, y estimular el trabajo en equipo.
    \item Participar en la gestión económica del centro.
    \item Coordinar, supervisar y controlar las diversas actividades desarrolladas en la zona.
    \item Presidir el consejo de salud de la zona básica de salud.
\end{itemize}

Por otra parte, el coordinador del equipo cuenta con la colaboración de un responsable de enfermería, con funciones de su supervisión y coordinación de las actividades de los profesionales de enfermería y unos responsables de programas de las áreas funcionales del equipo que son:

\begin{itemize}
    \item Área de Atención Directa
    \item Área de Docencia e Investigación
    \item Área de Administración
\end{itemize}

Las funciones del equipo de atención primaria vienen recogidas en la normativa que regula la organización funcional de las zonas de salud, entre las que cabe destacar:

\begin{itemize}
    \item Funciones de Salud Pública
    \item Funciones de Asistencia Sanitaria
    \item Funciones Docentes
    \item Funciones de Investigación
    \item Funciones Administrativas
    \item Funciones de Participación Comunitaria
\end{itemize}

Respecto a esta última, en el ámbito de cada Zona Básica de Salud se encuentra constituido el llamado Consejo de Salud, cuyo presidente es el propio Coordinador y donde están representados: Alcaldes, Asociaciones de Vecinos, Asociaciones de Consumidores, Ministerio de Educación y Ciencia, Sindicatos y otras asociaciones o grupos de ciudadanos legalmente constituidos y con fines de promoción comunitaria. De esta manera, el Consejo de Salud tiene la consideración de «Órgano de Participación Comunitaria» en las tareas de salud de cada Zona Básica.

Por otra parte, se contempla la existencia de algunos profesionales de Área en Atención Primaria para apoyar y complementar la labor de los Equipos de Atención Primaria.

Por otra parte, se contempla la existencia de algunos profesionales de Área en Atención Primaria para apoyar y complementar la labor de los Equipos de Atención Primaria. En este sentido, se han creado una serie de Unidades de Área que, por motivos de eficiencia, trabajan en más de una Zona Básica de Salud y son diferentes en cada zona, en función de criterios demográficos y demandas asistenciales. Las Unidades de Área son las siguientes.

\begin{itemize}
    \item Unidades de Fisioterapia
    \item Unidades de Salud Bucodental
    \item Unidades de Matronas
    \item Unidades de Pediatría de Área
    \item Unidades de Atención Urgente (PAC)
    \item Unidades de Atención a Domicilio (ESAD)
\end{itemize}

La oferta de servicios de Atención Primaria esta recogida en la cartera de servicios e incluye:

\begin{itemize}
    \item Servicios de atención al niño (vacunaciones infantiles, revisión del niño sano, prevención de la caries infantil y educación sanitaria en la escuela)
    \item Servicios de atención a la mujer (captación y seguimiento del emmabarazo, preparación al parto y visita posparto, vacunación de la rubeola, anticoncepción, prevención del cáncer ginecológico y atención en el climaterio)
    \item Servicios de atención al adulto-anciano (vacunaciones del adulto: gripe, tétanos, hepatitis B a grupos de riesgo; prevención de enfermedades cardiovasculares, atención a enfermos crónicos, atención domiciliaria a inmovilizados y terminales, prevención y detección de problemas en el anciano)
    \item Tratamientos fisioterapéuticos básicos y cirujía menor
\end{itemize}

Desde los equipos de atención primaria se presta atención sanitaria urgente las 24 horas del día, disponiendo para ello, especialmente en el medio urbano, de servicios de urgencia de refuerzo.

La implantación del modelo de Atención Primaria ha supuesto la modernización de los dispositivos del primer nivel asistencial (construcción de centros de salud y consultorios locales, mejora del equipamiento y de la tecnología médica, formación postgraduada y formación continuada de los médicos de familia, incorporación de profesionales sanitarios que refuerzan la oferta asistencial en Atención Primaria)

Ha supuesto también la integración de actividades de promoción y prevención de la salud y de cuidados de enfermería, la utilización sistemática de registros clínicos y el acceso a tecnología médicas.

En todos los centros de salud que atienden a una población superior a 10000 habitantes, existe una Unidad de Atención al Usuario, responsable del sistema de cita previa telefónica para el acceso a los servicios asistenciales, gestión de reclamaciones y sugerencias, gestión de prestaciones e información de usuario.

El Sistema de Gestión de Atención Primaria está basado en una estrategia de descentralización de las funciones de financiación y de provisión de servicios, según el cual las Áreas de Atención Primaria asumen la responsabilidad de la gestión de recursos y de los centros. Estos, a su vez, acuerdan con los equipos directivos de cada área un pacto de objetivos anual que incluye: cobertura de los servicios, cumplimiento de normas técnicas o criterios de calidad científico-técnica, objetivos de prescripción farmacológica, etc. El pacto de gestión incluye, igualmente, los presupuestos asignados al equipo para gastos de personal, farmacia, formación continuada de los profesionales además de compras y equipamiento.