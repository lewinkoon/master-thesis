En el siguiente capítulo se detalla la estructura organizativa del sistema sanitario de Castilla y León a lo largo de todos los estratos desde la conserjería de sanidad hasta el centro de salud de Laguna de Duero-

\section{Perfil de la organización}

El Área de Salud Valladolid Oeste (ASVAO) presta asistencia sanitaria a la población de aproximadamente la mitad de la provincia de Valladolid, la situada geográficamente en la zona oeste. Esta atención sanitaria se provee en sus dos niveles asistenciales: Atención Primaria (con 17 Zonas Básicas de Salud) y Atención Hospitalaria (con el Hospital Universitario Rio Hortega y el Centro de Especialidades de Arturo Eyries).

Esta organización es fruto del trabajo integrado de las Gerencias de Atención Primaria y Hospitalaria.

El Área de Salud Valladolid Oeste (ASVAO) forma parte del Sistema Público de Salud de Castilla y León, el cual comprende el conjunto de actuaciones y recursos públicos de la Administración Sanitaria de la Comunidad Autónoma y de las Corporaciones Locales, cuya finalidad es la promoción y protección de la salud en todos sus ámbitos, la prevención de la enfermedad, la asistencia sanitaria y la rehabilitación, todo ello bajo una perspectiva de asistencia sanitaria integral (Ley 8/2010, de 30 de agosto, de ordenación del sistema de salud de Castilla y León).

El Hospital Universitario Río Hortega fue inaugurado como Centro Hospitalario el 24 de Julio de 1953. En ese momento disponía de 310 camas y 72 cunas. Tras más de 50 años ubicado en pleno centro (cerca de la plaza de San Pablo), se trasladó a un nuevo edificio en el año 2008, esta vez situado a las afueras de la ciudad, en el barrio de las Delicias. En estos momentos cuenta con 640 camas instaladas y se configura como un hospital general, universitario y de tercer nivel. Además de prestar atención sanitaria al Área de Salud Valladolid Oeste, para determinadas prestaciones actúa como “Servicio de Referencia”, ampliando su cobertura a toda la provincia de Valladolid o incluso a varias provincias limítrofes. En algunos casos presta servicio a toda la Comunidad Autónoma.

En BOCYL de 24 de octubre de 2008, según ACUERDO 111/2008 de 23 de octubre de la Junta de Castilla y León, se reestructura el Área de Salud de Valladolid Oeste y el Área de Salud de Valladolid Este.

La apertura del nuevo Hospital Universitario Río Hortega, que cambia de ubicación, hace necesaria la adaptación del mapa sanitario del Área de Salud de Valladolid Oeste, pasando las Z.B.S. Centro Gamazo, Z.B.S. la Victoria, Z.B.S. Rural I, Z.B.S. Cigales a pertenecer al Área Este de Valladolid y las Z.B.S. Delicias I y Z.B .S. Delicias II pasan a pertenecer al Área Oeste de Valladolid.

Al frente de nuestra organización se han sucedido los siguientes nombramientos en los últimos años: Por Orden SAN/465/2012 de 18 de junio se nombró Gerente de Atención Especializada a D. Alfonso Montero Moreno, posteriormente por resolución de 15 de febrero de 2012 del Gerente Regional de Salud se acumularon las funciones de Gerente de Atención Primaria a las de Gerente de Atención Hospitalaria. El 18 de julio de 2015 fue nombrado como Gerente de Atención Primaria D. Eduardo García Prieto. Por último en el año 2018 a través de Orden SAN/901/2018 de 10 de agosto se nombró Gerente de Atención Especializada a D. José Miguel García Vela.

A partir del mes de marzo de 2017 todo el personal de la Gerencia de Atención Primaria se integró a trabajar en las instalaciones del HURH junto a de los diferentes servicios correspondientes.

\section{Cartera de servicios}

El Área de Salud Valladolid Oeste (ASVAO) es la organización a la que se adscribe todo el dispositivo sanitario del Área, compuesto por los 17 Centros de A. Primaria y por el Hospital Universitario Río Hortega. El objetivo primordial de esta organización es la prestación de asistencia sanitaria a la población del Área Oeste de Valladolid, tanto en Atención Primaria, enAtención Hospitalaria como en Atención Sociosanitaria (esta última compartida con la Gerencia Regional de Servicios Sociales).

La Atención Primaria es el nivel básico e inicial de atención, que garantiza la globalidad y continuidad de ésta a lo largo de la vida del paciente, actuando como gestor y coordinador de casos y regulador de flujos. Comprende actividades de promoción de la salud, educación sanitaria, prevención de la enfermedad, asistencia sanitaria, mantenimiento y recuperación de la salud, así como la rehabilitación física y el trabajo social.

La Atención Hospitalaria comprende las actividades asistenciales, diagnósticas, terapéuticas y de rehabilitación y cuidados, así como aquellas de promoción de la salud, educación sanitaria y prevención de la enfermedad, cuya naturaleza aconseja que se realicen en ese nivel. La Atención Hospitalaria garantizará la continuidad de la atención integral al paciente, una vez superadas las posibilidades de la Atención Primaria y hasta que aquel pueda reintegrarse en dicho nivel. Prestará además, servicios de hospitalización en régimen de internamiento, asistencia especializada en consultas, hospital de día (médico y quirúrgico), atención paliativa a enfermos terminales, salud mental y rehabilitación en pacientes con déficit funcional recuperable.

La Atención Sociosanitaria comprende el conjunto de cuidados destinados a aquellos enfermos, generalmente crónicos, que por sus especiales características pueden beneficiarse de la actuación simultánea y sinérgica de los servicios sanitarios y sociales para aumentar su autonomía, paliar sus limitaciones o sufrimientos y facilitar su reinserción social. En el ámbito sanitario, la Atención Sociosanitaria comprenderá los cuidados sanitarios de larga duración, la atención sanitaria a la convalecencia y la rehabilitación en pacientes con déficit funcional recuperable.

\subsection{Servicios en Atención Primaria}

La cartera de servicios en Atención Primaria de Valladolid Oeste está compuesta por:

\begin{table}[H]
    \centering
    \begin{tabular}{l}
        \toprule
        Servicios                                             \\
        \midrule
        Medicina familiar y comunitaria                       \\
        Pediatría                                             \\
        Enfermería                                            \\
        Unidad de salud bucodental                            \\
        Unidad de atención a la mujer                         \\
        Fisioterapia                                          \\
        Extracciones laboratorio                              \\
        Radiología                                            \\
        Urgencias                                             \\
        Asistemcia social                                     \\
        Cirugía menor                                         \\
        Diagnóstico ecográfico                                \\
        Farmacia                                              \\
        Unidad administrativa de citas y atención al paciente \\
        Prevención y promoción de la salud                    \\
        \bottomrule
    \end{tabular}
    \caption{Cartera de servicios en atención primaria}
\end{table}

Existe un amplio capítulo de actuaciones dirigidas a la Prevención y Promoción de la Salud, en el que se incluyen, entre otros, programas de vacunación infantil, programas de vacunación en el adulto, desarrollo de actividades preventivas en el adulto sano, prevención de obesidad infantil, atención a pacientes crónicos (hipertensión arterial, dislipemias, diabetes, EPOC, etc.), atención al paciente crónico pluripatológico complejo, prevención precoz de cáncer de mama, de colon, deshabituación tabáquica, atención a pacientes ancianos de riesgo o en situación terminal, violencia de género, etc.

\subsection{Servicios en Atención Especializada}

En la siguiente tabla se refleja la Cartera de Servicios Básica del Hospital Universitario Río Hortega.

\section{Mercado servido}

% Insertar pirámide de población actualizada

\section{Zonas Básicas de Salud}

Una Zona Básica de Salud está formada por un conjunto de profesionales sanitarios y no sanitarios, que constituyen el equipo de Atención Primaria, y que son los responsables de la prestación de la atención de la salud a la población en su demarcación sanitaria, y ello, de forma coordinada, integral, permanente y continuada, y orientada al individuo, a la comunidad y al medio ambiente.

Entre los integrantes del equipo se encuentran médicos de familia, pediatras, enfermeros, matronas, auxiliares de enfermería, trabajadores sociales, auxiliares administrativos y celadores. Además, integrando funcionalmente, existen veterinarios y farmacéuticos.

Cada zona de salud dispone de un centro de salud, estructura dotada de los medios necesarios para la prestación de las funciones que debe desempeñar el equipo de atención primaria. Además, del centro de salud, existen consultorios locales destinados a aquellas localidades de más de 50 habitantes, donde los profesionales sanitarios atienden la demanda asistencial bajo el criterio de una correcta accesibilidad de los servicios a la población.

El equipo de atención primaria se organiza jerárquicamente bajo la supervisión del coordinador del centro de salud, nombrado de entre los facultativos del equipo, responsable de la gestión de los recursos humanos y materiales. Entre sus funciones se encuentran las siguientes:

\begin{itemize}
    \item Asumir la representación oficial del equipo
    \item Ejercer la dirección y coordinación de todo el personal en lo referente al régimen interno de funcionamiento, así como resolver los conflictos en lo refenrente a dicho régimen interno de funcionamiento, y estimular el trabajo en equipo.
    \item Participar en la gestión económica del centro.
    \item Coordinar, supervisar y controlar las diversas actividades desarrolladas en la zona.
    \item Presidir el consejo de salud de la zona básica de salud.
\end{itemize}

Por otra parte, el coordinador del equipo cuenta con la colaboración de un responsable de enfermería, con funciones de su supervisión y coordinación de las actividades de los profesionales de enfermería y unos responsables de programas de las áreas funcionales del equipo que son:

\begin{itemize}
    \item Área de Atención Directa
    \item Área de Docencia e Investigación
    \item Área de Administración
\end{itemize}

Las funciones del equipo de atención primaria vienen recogidas en la normativa que regula la organización funcional de las zonas de salud, entre las que cabe destacar:

\begin{itemize}
    \item Funciones de Salud Pública
    \item Funciones de Asistencia Sanitaria
    \item Funciones Docentes
    \item Funciones de Investigación
    \item Funciones Administrativas
    \item Funciones de Participación Comunitaria
\end{itemize}

Respecto a esta última, en el ámbito de cada Zona Básica de Salud se encuentra constituido el llamado Consejo de Salud, cuyo presidente es el propio Coordinador y donde están representados: Alcaldes, Asociaciones de Vecinos, Asociaciones de Consumidores, Ministerio de Educación y Ciencia, Sindicatos y otras asociaciones o grupos de ciudadanos legalmente constituidos y con fines de promoción comunitaria. De esta manera, el Consejo de Salud tiene la consideración de «Órgano de Participación Comunitaria» en las tareas de salud de cada Zona Básica.

Por otra parte, se contempla la existencia de algunos profesionales de Área en Atención Primaria para apoyar y complementar la labor de los Equipos de Atención Primaria.

Por otra parte, se contempla la existencia de algunos profesionales de Área en Atención Primaria para apoyar y complementar la labor de los Equipos de Atención Primaria. En este sentido, se han creado una serie de Unidades de Área que, por motivos de eficiencia, trabajan en más de una Zona Básica de Salud y son diferentes en cada zona, en función de criterios demográficos y demandas asistenciales. Las Unidades de Área son las siguientes.

\begin{itemize}
    \item Unidades de Fisioterapia
    \item Unidades de Salud Bucodental
    \item Unidades de Matronas
    \item Unidades de Pediatría de Área
    \item Unidades de Atención Urgente (PAC)
    \item Unidades de Atención a Domicilio (ESAD)
\end{itemize}

La oferta de servicios de Atención Primaria esta recogida en la cartera de servicios e incluye:

\begin{itemize}
    \item Servicios de atención al niño (vacunaciones infantiles, revisión del niño sano, prevención de la caries infantil y educación sanitaria en la escuela)
    \item Servicios de atención a la mujer (captación y seguimiento del emmabarazo, preparación al parto y visita posparto, vacunación de la rubeola, anticoncepción, prevención del cáncer ginecológico y atención en el climaterio)
    \item Servicios de atención al adulto-anciano (vacunaciones del adulto: gripe, tétanos, hepatitis B a grupos de riesgo; prevención de enfermedades cardiovasculares, atención a enfermos crónicos, atención domiciliaria a inmovilizados y terminales, prevención y detección de problemas en el anciano)
    \item Tratamientos fisioterapéuticos básicos y cirujía menor
\end{itemize}

Desde los equipos de atención primaria se presta atención sanitaria urgente las 24 horas del día, disponiendo para ello, especialmente en el medio urbano, de servicios de urgencia de refuerzo.

La implantación del modelo de Atención Primaria ha supuesto la modernización de los dispositivos del primer nivel asistencial (construcción de centros de salud y consultorios locales, mejora del equipamiento y de la tecnología médica, formación postgraduada y formación continuada de los médicos de familia, incorporación de profesionales sanitarios que refuerzan la oferta asistencial en Atención Primaria)

Ha supuesto también la integración de actividades de promoción y prevención de la salud y de cuidados de enfermería, la utilización sistemática de registros clínicos y el acceso a tecnología médicas.

En todos los centros de salud que atienden a una población superior a 10000 habitantes, existe una Unidad de Atención al Usuario, responsable del sistema de cita previa telefónica para el acceso a los servicios asistenciales, gestión de reclamaciones y sugerencias, gestión de prestaciones e información de usuario.

El Sistema de Gestión de Atención Primaria está basado en una estrategia de descentralización de las funciones de financiación y de provisión de servicios, según el cual las Áreas de Atención Primaria asumen la responsabilidad de la gestión de recursos y de los centros. Estos, a su vez, acuerdan con los equipos directivos de cada área un pacto de objetivos anual que incluye: cobertura de los servicios, cumplimiento de normas técnicas o criterios de calidad científico-técnica, objetivos de prescripción farmacológica, etc. El pacto de gestión incluye, igualmente, los presupuestos asignados al equipo para gastos de personal, farmacia, formación continuada de los profesionales además de compras y equipamiento.