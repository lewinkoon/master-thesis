\documentclass[12pt, a4paper, twoside, openright]{report}

\usepackage[T1]{fontenc}
\usepackage[spanish]{babel}
\usepackage{parskip}

\usepackage{hyperref}
\usepackage{graphicx}
\usepackage{float}
\makenoidxglossaries

\newglossaryentry{sacyl-conecta}
{
    name={sacyl conecta},
    description={Aplicación de acceso a la gestión de citas con su Centro de Salud, documentación clínica y más información}
}

\newglossaryentry{medora}
{
    name={medora},
    description={Sistema informático que gestiona la historia clínica, la receta electrónica y otros procesos clínicos de Atención Primaria en la sanidad castellano leonesa}
}

\newglossaryentry{tarjeta-sanitaria}
{
    name={tarjeta sanitaria},
    description={Programa informático con el que se gestiona la base de datos de tarjetas sanitarias de los usuarios de Castilla y León}
}

\newglossaryentry{incapacidad-temporal}
{
    name={incapacidad temporal},
    description={Situación en la que se encuentran los trabajadores impedidos temporalmente para trabajar debido a enfermedad común o profesional y accidente, sea o no de trabajo, mientras reciban asistencia sanitaria de la Seguridad Social}
}

\newglossaryentry{electrocardiograma}
{
    name={electrocardiograma},
    description={Procedimiento simple, indoloro y rápido que registra la actividad eléctrica de su corazón}
}

\newglossaryentry{guardia}
{
    name={guardia médica},
    description={Jornada laboral de 24 horas que realiza un facultativo en centros hospitalarios para garantizar la asistencia a los pacientes}
}

\newglossaryentry{matrona}
{
    name={matrona},
    description={Profesional sanitario que acompaña a la mujer en su embarazo, parto y posparto}
}

\newglossaryentry{ortopantomografías}
{
    name={ortopantomografías},
    description={Técnica de radiografía panorámica que se sirve de rayos X para ofrecer información detallada acerca de
    las estructuras dentales y la anatomía oral}
}

\newglossaryentry{volante}
{
    name={volante},
    description={Documento impreso que el profesional de la sanidad utiliza para prescribir pruebas diagnósticas, tratamientos, ingresos hospitalarios o intervenciones quirúrgicas}
}

\newglossaryentry{nodulos-tiroideos}
{
    name={nódulos tiroideos},
    description={Bultos sólidos o llenos de líquido que se forman dentro de la tiroides, una glándula pequeña ubicada en la base del cuello, justo encima del esternón}
}

\newglossaryentry{sintrom}
{
    name={sintrom},
    description={Medición de la determinación del INR (Razón Normalizada Internacional) o lo que es lo mismo, saber
    cuánto tarda una muestra de sangre en formar un coágulo}
}

\begin{document}

\title{Aplicación de técnicas de mejora continua en el centro de salud de Laguna de Duero}
\author{Luis Llamas Fernández}
\date{\today}
\maketitle

\thispagestyle{empty}

\chapter*{Resumen}
Este trabajo muestra un análisis de los distintos flujos de información y procesos que ocurren dentro de la sección 
administrativa del Centro de Salud de Laguna de Duero.
Para ello, se han hecho uso de técnicas de mejora continua y Lean 
Management, como la herramienta de los cinco porqués.
El estudio tiene como objetivo estandarizar las tareas que se realizan 
los distintos puestos además de descubrir las ineficiencias y, consecuentemente, plantear mejoras que ayuden a ofrecer un 
mejor servicio.
\keywords{sanidad, lean, bpmn, cinco porqués}

\setcounter{page}{1}

\chapter*{Abstract}
This work shows an analysis of the different information flows and processes that occur within the administrative section of the Laguna de Duero Health Centre.
For this purpose, we have made use of continuous improvement techniques and Lean  Management, such as the five whys tool.
The aim of the study is to standardise the tasks performed at the different posts, as well as to discover inefficiencies and, consequently, to propose improvements that will help to offer a better service.
\keywords{healthcare, lean, bpmn, five whys}

\chapter*{Agradecimientos}
Quiero agradecer a todos los profesores que he tenido, tanto en la universidad como fuera de ella, por haber fomentado, aunque algunos más que otros, el desarrollo de mi curiosidad. También quiero agradecer a aquellos que me han formado como profesional y como persona, realizando un trabajo que nunca se podrá valorar lo suficiente. También a mi familia, que me ha dedicado todo su tiempo, todo su esfuerzo y todos sus recursos con tal de educarme y formarme lo mejor posible para afrontar la vida.

\tableofcontents

\chapter{Introducción}
\section{Contexto}

En los últimos años, tras el reciente periodo de pandemia y la situación económica del país, ha quedado patente la necesidad de una buena gestión sanitaria por parte de la administración con el fin de atender la creciente demanda de servicios asistenciales por parte de la población haciendo un uso eficiente de los recursos sanitarios.

La correcta gestión de los recursos en la sanidad es un tema que en menor o mayor grado concierne a todos los ciudadanos por un motivo principal: el derecho a la protección de la salud. Este se reconoce en el artículo 43 de la Constitución y se concreta en la Ley General de Sanidad (Ley 14/1986), que establece su financiación pública, universalidad y gratuidad; su descentralización autonómica y su integración en el Sistema Nacional de Salud (SNS). En definitiva, todas las personas tienen derecho a una atención sanitaria de calidad en condiciones de igualdad.

En base a esta premisa, recientemente se han sucedido numerosas manifestaciones en defensa de la sanidad pública por parte de distintas asociaciones y representantes que proponen un aumento de las partidas destinadas a equipos y personal del sistema sanitario público. Este trabajo va un paso más allá, presentando distinas propuestas en base a técnicas de mejora continua que mejoren la eficiencia de los recursos ya disponibles, tanto económicos como de personal.

\section{Motivación personal}

El sector sanitario es un campo en el que no es común encontrar a un ingeniero industrial. Y cuando esto sucede, normalmente se suele asociar a aspectos técnicos como pueden ser el mantenimiento de las instalaciones térmicas y eléctricas o la ingeniería biomédica. Más extraño es, encontrar a un ingeniero involucrado en la mejora de procesos asistenciales.

Fue lo exótico de este campo junto con la posibilidad de poder dedicar mi tiempo a ayudar a los demás, aunque sea de forma indirecta, en un sector tan social como es el sanitario lo que me llevó a realizar este trabajo fin de máster.

Aunque la ingeniería de procesos, al menos tal y como se imparte en la carrera, está orientada principalmente a la industria manufacturera en cuanto a producción de bienes de equipo, todas las metodologías aprendidas, kaizen, 5s, JIT o kanban son muchas veces aplicables a los procesos asistenciales.

\section{Objetivos}

El objetivo principal de este trabajo fin de máster es realizar una lista estandarizada y clasificada por tipo de proceso de todas las tareas administrativas que se realizan en la seccióna administrativa del centro de salud de Laguna de Duero.

Para conseguir este objetivo se plantean los siguientes subobjetivos.

\begin{enumerate}
    \item Crear un grupo de trabajo con los representantes del personal administrativo del centro.
    \item Realizar una lista de tareas comunes al personal administrativo tanto de ventanilla como de interior
    \item Elaborar una tabla con los problemas de mayor importancia y sus respectivas causas
    \item Plantear mejoras en los distintos procesos que se llevan a cabo
\end{enumerate}

\section{Alcance}

El área de estudio comprende la sección administrativa del centro de salud de Laguna de Duero (Valladolid) perteneciente al Área de Salud Valladolid Oeste. La decisión de elegir este centro fue motivada por la gran variedad de servicios que ofrece: extracciones, salud bucodental, fisioterapia, radiología y un punto de atención continuada. Además, el centro se encuentra a medio camino entre el ámbito rural y urbano haciendolo idóneo para realizar este proyecto en vista de aplicarlo al resto de centros de salud del área.

Para llevar a cabo el estudio se han hecho uso de distintas técnicas lean pertenecientes la ingeniería de procesos:

\begin{itemize}
    \item Estandarización de tareas
    \item Diagrama de procesos BMPN
    \item Cuadro de los cinco porqués
\end{itemize}

Dado que este proyecto se ha propuesto desde la Dirección de Gestión de la Gerencia de Atención Primaria, solamente se ha involucrado al personal administrativo. Si se cumplen los objetivos, en el futuro se pretende incluir en el grupo de trabajo a representantes de los demás grupos de interés involucrados en los procesos del centro de salud: médicos, enfermeras, auxiliares de enfermería y celadores entre otros.

\section{Estructura memoria}

Esta memoria sigue la estructura especificada en la guía docente para la asignatura TFM del Máster en Ingeniería Industrial de la Escuela de Ingenieros Industriales (Universidad de Valladolid).

Siguiendo al presente capítulo de introducción, donde se expone el contexto, la motivación, el alcance y los objetivos de este trabajo, se encuentran los siguientes capítulos:

\begin{itemize}
    \item Capítulo 2. Atención Primaria en el Área de Salud Valladolid Oeste
    \item Capítulo 3. Herramientas de mejora continua en el sector sanitario
    \item Capítulo 4. Desarrollo del proyecto
    \item Capítulo 5. Estudio económico
    \item Capítulo 6. Conclusiones
\end{itemize}

Finalmente se adjunta las referencias bibliográficas utilizadas.


\chapter{Centro de Salud de Laguna de Duero}
\input{capitulos/2-centro-salud}

\chapter{Metodología lean en sanidad}
\input{capitulos/3-metodologia.tex}

\chapter{Desarrollo del proyecto}
En este capítulo se explica cómo ha sido la dinámica del trabajo a la hora de analizar las tareas administrativas del centro además de desarrollar en detalle todos los procesos que se llevan a cabo en la oficina.

\section{Distribución de la plantilla}

El equipo administrativo del centro de salud está 

\section{Procesos administrativos}

En este apartado se describen los distintos procesos que existen en la sección administrativa. 

% \subsection{Agendas}

% EL proceso de realización de agendas consiste en la asignación de tareas sanitarias en la distintas franjas de tiempo de la jornada laboral. Las agendas están orientadas para el personal sanitario que pasan consulta o realizan algún tipo de prueba al paciente: médicos, enfermeras, fisioterapeutas, trabajadores sociales, etc.

% Poniendo de ejemplo un médico facultativo, 

\subsection{Cargos a terceros}

Por lo general, los pacientes que acuden al centro de salud tienen acceso gratuito a la cartera de servicios de Atención Primaria, sin embargo, hay ciertos casos en los que el Sistema Público de Salud no asume el coste y que por tanto se debe de cobrar al paciente.
Estos casos pueden ser:

\subsubsection{Convenios internacionales}

En este caso de debe rellenar el formulario \textit{H-111}, indicando diagnóstico, fecha de asistencia y firma del paciente.
Es muy importante presentar una fotocopia de la tarjeta sanitaria europea o un documento equivalente.

En el caso de que no se presente, se debe anotar el número de teléfono y el correo electrónico para poder contactar con el paciente desde la sección de ``Cargos a terceros'' del Hospital Universitario Rio Hortega.

También es necesario que el paciente firme la nota informativa ya que es el documento requerido para la protección de datos y el compromiso de hacerse cargo del importe del gasto.

\subsubsection{Accidentes de tráfico}

En el caso de un accidente de tráfico es necesario que el paciente rellene la ficha de asistencia sanitaria, firme la nota informativa y complete la solicitud de datos de accidente de tráfico.
Este último documento debe concordar con los datos presentes en el atestado del accidente.

\subsubsection{Accidentes de trabajo}

En este caso tan solo es necesario rellenar la ficha de asistencia sanitaria y firmar la nota informativa.

\subsubsection{Espectáculos taurinos}

Se rellena la ficha de asistencia sanitaria junto con la firma de la nota interior.
En este caso, se debe de completar el documento ``Lesionados en espectáculos taurinos'' especificando el lugar donde se realiza el espectáculo y quién es el responsable del mismo (Ayuntamiento, asociaciones taurinas, etc).

\subsubsection{Agresiones}

En este caso tan solo es necesario rellenar la ficha de asistencia sanitaria y firmar la nota informativa.

\subsubsection{Accidentes deportivos o de caza}

Además de la ficha de asistencia sanitaria y la nota informativa, el paciente debe presentar una fotocopia de su licencia federativa junto con la información del seguro que cubre el siniestro, especificando el número de póliza y el número de siniestro.

\subsubsection{Seguros privados}

Se rellena la ficha de asistencia sanitaria junto con la firma de la nota informativa.
Se debe presentar también una copia del parte de accidente, la petición de asistencia o cualquier otro documento de acepctación del gasto que haya sido expedido por la empresa, la compañía de seguros o la mutua.
En el caso de que sea un paciente desplazado, se debe presentar la fotocopia de la tarjeta de mutualista.

\subsubsection{Particulares}

Se completa la ficha de asistencia sanitaria y se firma la nota informativa. Este último documento es muy importante ya que hace constar que el paciente se responsabiliza del importe del cargo.

\subsection{Reclamaciones}

Se entregarán al usuario siempre que lo solicite. El usuario deberá identificarse con su documento de identidad y el personal administrativo cumplimentará la hoja elaborada a tal efecto, donde se encuentran las hojas de reclamación.

Una vez entregada por el usuario, se registra de entrada y se entrega la hoja amarilla al paciente. Las dos hojas restantes se quedan en administración.

\subsection{Gestión de citas}

Un paciente puede solicitar una cita con un profesional sanitario de atención primaria por dos vías: en el propio centro de salud o a través de la aplicación de SACYL Conecta.

Si el usuario decide solicitar cita previa vía online la realización de la tarea es diferida, no tiene por qué realizarse en el momento. De esta forma, al final de la jornada o cuando el trabajo lo crea oportuno, se revisan las peticiones de citas y se tramitan en el programa Medora.

\subsection{Gestión de la tarjeta sanitaria}

Tanto la creación de una nueva tarjeta sanitaria como la modificación de algunos de sus parámetros es otro de los procesos más demandados por la población. Ambas operaciones se realizan a través del programa \textit{Tarjeta Sanitaria} que actúa como base de datos de las datos identificativos de los usuarios asignados a la Zona Básica de Salud a la que pertenece el centro, en este caso Laguna de Duero.

Se pueden editar multitud de campos, pero las operaciones más demandadas son las siguientes:

\begin{enumerate}
    \item Creación de tarjeta sanitaria
    \item Cambio de médico de familia
    \item Cambio de consultorio o de centro de salud (conlleva cambio de médico de familia)
\end{enumerate}

\subsection{Información general}

Una de las funciones de los puestos administrativos de ventanilla es la de informar a los paciente y entregar justificantes de diversos tipos. Aunque las solicitudes en este proceso son muy diversas, en general se pueden distinguir tres tipos:

\begin{itemize}
    \item Utilización de la aplicación de SACYL Conecta
    \item Impresión del parte de baja laboral
    \item Impresión de justificante COVID
\end{itemize}

\subsection{Tramitación de permisos retribuidos}

Otro de los procesos que más tiempo y esfuerzo administrativo requiere son las tramitaciones de permisos retribuidos del personal del centro. Esto incluye a médicos, enfermeras, auxiliares, técnicos de rayos, celadores y a los propios auxiliares administrativos. Más adelante, en el capítulo de propuestas de mejora, se hablará de la implantación del programa AIDA de gestión de personal que elimina todos los trámites en papel y el envío físico de los justificantes a los distintos aprobadores.

Sin embargo, a día de hoy, el flujo de información sigue siendo manual.

\begin{enumerate}
    \item \textbf{Recepción de solicitudes de permiso}. En cualquier momento del día se pueden recibir solicitudes de permisos por parte del personal. Pueden estar o no aprobadas por el jefe de servicio correspondiente (coordinador médico o la supervisora de enfermería). 
    \item Envío de 
\end{enumerate}

\section{Análisis causa raíz}

\chapter{Estudio económico}
En este capítulo se desarollará el estudio económico del proyecto llevado a cabo en el Centro de Salud de Laguna de Duero.
Para ello, se desglosará su elaboración en las distintas fases que lo componen, desde la propuesta de realización hasta la redacción del informe final.
Para su valoración, se tienen en cuenta las horas dedicadas de los distintos participantes además de los recursos tanto directos como indirectos empleados en las distintas fases.

\section{Profesionales involucrados}

Para la realización de este proyecto ha sido necesaria la participación de distintos profesionales del sector sanitario.
En primer lugar, se encuentra el Subdirector de Gestión y SSGG del Área de Salud Valladolid Oeste que actúa de promotor del proyecto.
Es el responsable de que el proyecto se lleve a cabo además de coordinar y aconsejar en los distintas situaciones.
Será el encargado de la aprobación y validación final del proyecto.

Seguidamente, se encuentra el Ingeniero Industrial que llevará a cabo toda las tareas de recopilación de información y documentación del trabajo.
Además, se encarga de organizar y formar el grupo de trabajo del centro.
A partir de sus conocimientos, sobre todo, en organización industrial, realizará las tareas de análisis y síntesis de los procesos principales de la sección administrativa del centro.

Finalmente, se encuentra los administrativos que forman el grupo de trabajo del proyecto.
Se encargan de ofrecer toda la información necesaria sobre sus labores además de exponer las distintas dificultades con las que se encuentran en su día a día.
Además, pueden proponer distintas mejoras y soluciones a las problemáticas analizadas en el propio proyecto.

\section{Fases del proyecto}

Conocer las fases del proyecto es necesario para estimar los costes económicos asociados a cada fase.
A continuación, se describen las fases en las que se ha desarollado el proyecto.
Además, en la Figura~\ref{fig:gantt} se muestra de forma visual la planificación del proyecto en base a un diagrama de Gantt.

\begin{figure}[H]
    \centering
    \includegraphics[width=\textwidth]{img/gantt-diagram.png}
    \caption{Diagrama de Gantt del proyecto}
    \label{fig:gantt}
\end{figure}

\subsection{Propuesta de elaboración de proyecto}

El proyecto comienza con la propuesta por parte de la Subdirección de Gestión del Área de Salud Valladolid Oeste de la realización de un proyecto de aplicación de técnicas de mejora continua a un centro de Atención Primaria.
Es en esta fase donde se elige el centro de Laguna de Duero por ser el mejor representante de los servicios ofrecidos en las distintas Zonas Básicas de Salud.
Además, se exponen los objetivos que se quieren alcanzar, la estandarización de tareas junto con la propuesta de mejoras, junto con la metodología a seguir.

\subsection{Formación del grupo de trabajo}

En esta fase se forma el grupo de trabajo con el que se desarrollará todo el proyecto.
Previamente, fue necesario documentarse en la aplicación de técnicas de mejora continua en entornos de oficina y sanitarios.
Además, en esta fase se preparó una página web en \Gls{sharepoint} que sirvió para poder compartir con todos los miembros del grupo el progreso del trabajo.

\subsection{Recopilación de información}

Una vez formado el grupo, se organizaron reuniones semanales con un grupo pequeño de trabajadores a modo de muestra que conocieran bien su puesto de trabajo para poder recopilar toda la información de las operaciones y tareas llevadas a cabo en la sección.
Se realizaron un total de seis reuniones que se pueden separar en dos fases.
Por un parte las cuatro primeras reuniones se centraron en recoger información de todas las tareas administrativas que se realizan en los puestos administrativos de cara a definir un estándar y, por otras parte, las dos últimas reuniones se centraron en obtener un listado de problemas su posterior análisis de las causas raíz con la técnica de los cinco porqués.

\subsubsection{Reunión 1: Toma de contacto}

En esta primera reunión asistieron los dos subdirectores de gestión del área.
Fue una presentación corta en la que se explicó al grupo de trabajo la necesidad de realizar una estandarización de las tareas administrativas de la sección del centro con el fin de analizar las cargas de trabajo reales.
Además, se propuso el análisis mediante la técnica de los cinco porqués las causas raíz de los problemas administrativos.

\subsubsection{Reunión 2: Citas previas y tarjeta sanitaria}

Dado que la gestión de citas previas era proceso principal de la sección administrativa fue el primero que se trató.
Se explicó con detalle los distintos tipos de citas: consultas generales, pruebas de laboratorio y citas de especialidades.
Además de explicar el proceso, se entregó una copia a modo de ejemplo de los volantes de extracciones y de radiología en los cuales el facultativo especifica qué pruebas concretas se deben de realizar al paciente.

\subsubsection{Reunión 3: Agendas, cargos a terceros, reclamaciones y absentismo}

En esta reunión se trataron los procesos de gestión de agendas, la tramitación de cargos a terceros, reclamaciones y las solicitudes de permisos retribuidos.
En cada uno de ello se ofrecieron los distintos formularios utilizados en cada proceso.

\subsubsection{Reunión 4: Tareas generales}

En esta reunión sobre la estandarización se desarrollaron por parte del personal otras tareas generales que no pertenecen a ninguno de los procesos principales.

\subsubsection{Reunión 5: Listado de problemas para los cinco porqués}

En esta reunión se realizó la presentación y explicación de la herramienta de los cinco porqués con el fin de analizar las causas raíz de los problemas presentes en la sección.
Se realizó un ``brainstorming'' en el que participaron todos los miembros del grupo de trabajo con el fin de definir todos los problemas administrativos.
Posteriormente se filtraron para valorar los que más impacto tenían en el rendimiento del equipo.

\subsubsection{Reunión 6: Análisis de las causas raíz y propuestas de mejora}

En esta última reunión se fueron comentando cada uno de los problemas listados en la anterior reunión a la vez que se aplicaba la ténica de los cinco porqués.
Una vez obtenidos las causas de cada problema, se propusieron posibles soluciones a cada una de las problemáticas.
Finalmente, se discutió la viabilidad de cada una.

\subsection{Redacción de la documentación}

Finalmente, tras recoger todos los datos necesarios se procede a analizar toda la información recopilada y a redactar la memoria del trabajo.

\section{Estimación de costes}

En este apartado se lleva a cabo la valoración económica del proyecto a partir de todos los recursos necesarios para su desarrollo.
En la evaluación económica se tienen en cuenta una relación de los costes asociados a los siguientes apartados: personal, amortizaciones de los  equipos informáticos, materiales consumibles y servicios indirectos del proyecto.
Se analiza cada una de esas partes de forma individual con el objetivo de conocer la influencia que tiene cada una de ellas sobre el valor final del estudio.

\subsection{Horas efectivas anuales y tasas horarias de personal}

En primer lugar obtienen las horas trabajadas anuales de los profesionales implicados.
Dado que todos los profesionales pertenecen al Sacyl las horas trabajadas se corresponden con un total de 35 horas semanales \cite{noauthor_orden_2022}, es decir, \textbf{1533 horas} anuales.
Seguidamente, se calculan los costes horarios (Tabla~\ref{tab:salarios}) acorde a las tablas salariales del año 2023 del Sacyl~\cite{noauthor_orden_2023}.

\begin{table}[H]
    \centering
    \begin{tabular}{llll}
        \toprule
        Concepto                 & Sub. Gestión & Ing. Industrial & Aux. Administativo \\
        \midrule
        Salario                  & € 56,369.32  & € 38,938.06     & € 21,898.88        \\
        Seguridad Social Empresa & € 19,729.26  & € 13,628.32     & € 7,664.61         \\
        \midrule
        Total                    & € 76,098.58  & € 52,566.38     & € 29,563.49        \\
        Coste horario            & € 48.22      & € 33.31         & € 18.73            \\
        \bottomrule
    \end{tabular}
    \caption{Coste salarial de los profesionales participantes}
    \label{tab:salarios}
\end{table}

\subsection{Costes de amortización de equipos informáticos}

En este apartado se han calculado los costes de las amortizaciones de los equipos utilizados en el proyecto.
La amortización es la pérdida de valor que los bienes del inmovilizado de la empresa sufren a medida que se van usando, porque se producen avances con la investigación que hacen que se vayan quedando anticuados o por el mero transcurso del tiempo.
En este caso fue necesario el uso de un ordenador junto con un conjunto de periféricos.
También se incluyen las licencias de los distintos programas informáticos utilizados.

La amortización de estos equipos se ha calculado considerando una vida útil de cinco años tal y como se muestra en las Tablas~\ref{tab:coste-equipos} y \ref{tab:coste-amortizacion}.

\begin{table}[H]
    \centering
    \begin{tabular}{llll}
        \toprule
        Equipo              & Coste      & Cantidad & Coste total \\
        \midrule
        Ordenador HP        & € 783.34   & 1        & € 783.34    \\
        Licencia Windows 10 & € 108.23   & 1        & € 108.23    \\
        Licencia Office 365 & € 127.12   & 1        & € 127.12    \\
        Impresora Kyocera   & € 97.23    & 1        & € 97.23     \\
        \midrule
        Total a amortizar   & € 1,115.92 &          & € 1,115.92  \\
        \bottomrule
    \end{tabular}
    \caption{Amortización anual en euros de los equipos utilizados}
    \label{tab:coste-equipos}
\end{table}

\begin{table}[H]
    \centering
    \begin{tabular}{lll}
        \toprule
        Tipo    & Número  & Amortización \\
        \midrule
        Diaria  & € 3.06  & € 0.61       \\
        Semanal & € 21.46 & € 4.29       \\
        Horaria & € 0.71  & € 0.14       \\
        \bottomrule
    \end{tabular}
    \caption{Amortización anual en euros de los equipos utilizados}
    \label{tab:coste-amortizacion}
\end{table}

\subsection{Coste de material consumible}

El coste de los consumibles (papeles de impresora, disquetes, CD's, etc.) se ha calculado según su consumo medio, por persona y hora de trabajo como se muestra en la Tabla~\ref{tab:coste-consumibles}.

\begin{table}[H]
    \centering
    \begin{tabular}{ll}
        \toprule
        Consumible                & Coste    \\
        \midrule
        Papel de impresión        & € 63.00  \\
        Suministro de impresión   & € 273.00 \\
        CD y USB                  & € 16.00  \\
        Otros                     & € 37.00  \\
        \midrule
        Coste anual por persona   & € 389.00 \\
        Coste horario por persona & € 0.25   \\
        \bottomrule
    \end{tabular}
    \caption{Costes de material consumible}
    \label{tab:coste-consumibles}
\end{table}

\subsection{Costes indirectos}

El coste indirecto es aquel que afecta al proyecto pero que no puede imputarse a ninguna de las fases o etapas del mismo.
En este caso se hacen referencia a los consumos de electricidad, telefonía, etc.
En la Tabla~\ref{tab:coste-indirecto} se muestran los costes por persona y hora para cada uno de los conceptos.

\begin{table}[H]
    \centering
    \begin{tabular}{ll}
        \toprule
        Concepto                  & Coste    \\
        \midrule
        Teléfono                  & € 63.00  \\
        Electricidad              & € 273.00 \\
        Otros                     & € 146.00 \\
        \midrule
        Coste annual por persona  & € 482.00 \\
        Coste horario por persona & € 0.31   \\
        \bottomrule
    \end{tabular}
    \caption{Costes indirecto}
    \label{tab:coste-indirecto}
\end{table}

\subsection{Tiempos asociados a cada fase del proyecto}

Los costes de personal corresponden a todas las personas implicadas en el proyecto que han dedicado parte de su jornada laboral al mismo.
En la Tabla~\ref{tab:horas-trabajadas} se muestra el reparto de horas trabajadas por cada grupo de personal y por fase de proyecto.
Cabe destacar que en la columna de administrativos se ha tenido en cuenta las horas dedicadas por los cuatro trabajadores de la sección administrativa del centro de salud.

\begin{table}[H]
    \centering
    \begin{tabular}{llllll}
        \toprule
        Personal            & Fase I & Fase II & Fase III & Fase IV & Total \\
        \midrule
        Sub. Gestión        & 1      & 0       & 12       & 0       & 13    \\
        Ing. Industrial     & 8      & 32      & 48       & 212     & 300   \\
        Aux. Administrativo & 0      & 0       & 48       & 0       & 48    \\
        \midrule
        Total               & 9      & 32      & 108      & 212     & 361   \\
        \bottomrule
    \end{tabular}
    \caption{Desglose de horas dedicadas por tipo de profesional y fase}
    \label{tab:horas-trabajadas}
\end{table}

\subsection{Costes asignados a cada fase del proyecto}

Para calcular los costes asignados a cada fase se tendrán en cuenta todos los costes de amortización, consumibles, indirectos y de personal junto con las horas dedicadas en cada etapa por todos los profesionales.

\subsubsection{Fase I: Propuesta de elaboración de proyecto}

En esta primera fase tan solo intervienen el Subdirector y el Ingeniero Industrial con el fin de proponer y establecer los objetivos.

\begin{table}[H]
    \centering
    \begin{tabular}{@{}llll@{}}
        \toprule
        Concepto            & Horas & Coste horario & Coste total \\ \midrule
        Sub. Gestión        & 1     & € 48.22       & € 48.22     \\
        Ing. Industrial     & 8     & € 33.31       & € 266.50    \\
        Aux. Administrativo & 0     & € 33.31       & € -         \\
        Amortización        & 197   & € 0.14        & € 27.82     \\
        Consumibles         & 39    & € 0.25        & € 9.70      \\
        Costes indirectos   & 39    & € 0.31        & € 12.02     \\
        \midrule
        Coste total         &       &               & € 364.26    \\
        \bottomrule
    \end{tabular}
    \caption{Costes asociados a la Fase I}
    \label{tab:fase-propuesta}
\end{table}

\subsubsection{Fase II: Formación del grupo de trabajo}

En esta segunda fase interviene solamente el Ingeniero Industrial en su labor de preparación del grupo y de estudio de la aplicación de las técnicas de mejora continua en el ámbito sanitario.

\begin{table}[H]
    \centering
    \begin{tabular}{llll}
        \toprule
        Concepto            & Horas & Coste horario & Coste total \\
        \midrule
        Sub. Gestión        & 0     & € 48.22       & € -         \\
        Ing. Industrial     & 32    & € 33.31       & € 1,065.98  \\
        Aux. Administrativo & 0     & € 33.31       & € -         \\
        Amortización        & 0     & € 0.14        & € -         \\
        Consumibles         & 140   & € 0.25        & € 34.48     \\
        Costes indirectos   & 140   & € 0.31        & € 42.73     \\
        \midrule
        Coste total         &       &               & € 1,143.19  \\
        \bottomrule
    \end{tabular}
    \caption{Costes asociados a la Fase II}
    \label{tab:fase-formacion}
\end{table}

\subsubsection{Fase III: Recopilación de información}

En esta tercera intervienen el Ingeniero Industrial junto con los administrativos del grupo de trabajo. En las horas se tienen en cuentan todas las reuniones organizadas.

\begin{table}[H]
    \centering
    \begin{tabular}{llll}
        \toprule
        Concepto            & Horas & Coste horario & Coste total \\
        \midrule
        Sub. Gestión        & 12    & € 48.22       & € 578.70    \\
        Ing. Industrial     & 48    & € 33.31       & € 1,598.98  \\
        Aux. Administrativo & 48    & € 33.31       & € 1,598.98  \\
        Amortización        & 0     & € 0.14        & € -         \\
        Consumibles         & 472   & € 0.25        & € 116.38    \\
        Costes indirectos   & 472   & € 0.31        & € 144.20    \\
        \midrule
        Coste total         &       &               & € 4,037.23  \\
        \bottomrule
    \end{tabular}
    \caption{Costes asociados a la Fase III}
    \label{tab:fase-recopilacion}
\end{table}

\subsubsection{Fase IV: Redacción de la documentación}

En esta cuarta, y última, fase interviene solamente el Ingeniero Industrial que se encarga de analizar toda la información recopilada y redactar el informe final.

\begin{table}[H]
    \centering
    \begin{tabular}{llll}
        \toprule
        Concepto            & Horas & Coste horario & Coste total \\
        \midrule
        Sub. Gestión        & 0     & € 48.22       & € -         \\
        Ing. Industrial     & 212   & € 33.31       & € 7,062.15  \\
        Aux. Administrativo & 0     & € 33.31       & € -         \\
        Amortización        & 0     & € 0.14        & € -         \\
        Consumibles         & 927   & € 0.25        & € 228.44    \\
        Costes indirectos   & 927   & € 0.31        & € 283.06    \\
        \midrule
        Coste total         &       &               & € 7,573.65  \\
        \bottomrule
    \end{tabular}
    \caption{Costes asociados a la Fase IV}
    \label{tab:fase-redaccion}
\end{table}

\section{Cálculo del coste total}

La estimación del coste total de trabajo de fin de máster se obtiene de la suma de los distintos costes totales de cada una de las partidas definidas en los apartados anteriores.
En la Tabla~\ref{tab:coste-total} se reflejan todos los costes y su peso sobre el total.

\begin{table}[H]
    \centering
    \begin{tabular}{lll}
        \toprule
        Actividad                                       & Horas & Coste       \\
        \midrule
        Fase I: Propuesta de elaboración del   proyecto & 9     & € 364.26    \\
        Fase II: Formación del grupo de trabajo         & 32    & € 1,143.19  \\
        Fase III: Recopilación de información           & 108   & € 4,037.23  \\
        Fase IV: Redacción de la documentación          & 212   & € 7,573.65  \\
        \midrule
        Total                                           & 361   & € 13,118.33 \\
        \bottomrule
    \end{tabular}
    \caption{Coste total del proyecto}
    \label{tab:coste-total}
\end{table}

A estos costes habría que sumar impuestos indirectos como el IVA y el margen comercial de beneficio.

\chapter{Conlusiones}
En este último capítulo se plantean las conclusiones del proyecto y el trabajo a futuro que sería conveniente realizar en este campo.

\appendix
\chapter{Diagramas BPMN de los procesos }

\begin{figure}
    \centering
    \begin{sideways}
        \includegraphics[width=0.95\textheight]{img/proceso-consultas.png}
    \end{sideways}
    \caption{Diagrama de proceso de consultas}
    \label{fig:proceso-consultas}
\end{figure}

\begin{figure}
    \centering
    \begin{sideways}
        \includegraphics[width=0.95\textheight]{img/proceso-tarjeta.png}
    \end{sideways}
    \caption{Diagrama de proceso de alta o modificación de tarjeta sanitaria}
    \label{fig:proceso-tarjeta}
\end{figure}

\begin{figure}
    \centering
    \begin{sideways}
        \includegraphics[width=0.95\textheight]{img/proceso-reclamaciones.png}
    \end{sideways}
    \caption{Diagrama de proceso de reclamaciones}
    \label{fig:proceso-reclamaciones}
\end{figure}

\begin{figure}
    \centering
    \begin{sideways}
        \includegraphics[width=0.95\textheight]{img/proceso-permisos.png}
    \end{sideways}
    \caption{Diagrama de proceso de permisos retribuidos}
    \label{fig:proceso-permisos}
\end{figure}

\begin{figure}
    \centering
    \begin{sideways}
        \includegraphics[width=0.95\textheight]{img/proceso-cargos.png}
    \end{sideways}
    \caption{Diagrama de proceso de cargos a terceros}
    \label{fig:proceso-cargos}
\end{figure}

\begin{figure}
    \centering
    \begin{sideways}
        \includegraphics[width=0.95\textheight]{img/proceso-agendas.png}
    \end{sideways}
    \caption{Diagrama de proceso de gestión de agendas}
    \label{fig:proceso-agendas}
\end{figure}

\printnoidxglossaries

\printbibliography

\end{document}