\documentclass[12pt, a4paper]{report}
\usepackage[math]{iwona}
\usepackage[T1]{fontenc}

\usepackage[T1]{fontenc}
\usepackage[spanish]{babel}
\usepackage{parskip}

\usepackage{hyperref}
\usepackage{graphicx}
\usepackage{float}

\begin{document}

\title{Aplicación de técnicas de mejora continua en el centro de salud de Laguna de Duero}
\author{Luis Llamas Fernández}
\date{\today}
\maketitle

\chapter*{Resumen}
GIS es una potente herramienta que puede utilizarse para localizar manantiales originados en ofiolitas. Las características únicas asociadas a estos manantiales incluyen un entorno subsuperficial reductor que reacciona a bajas temperaturas produciendo fluidos de formación de alto pH, ricos en Ca y con un alto contenido en hidrógeno y metano disueltos. Debido a sus características químicas únicas, estas zonas suelen estar asociadas a microbios y se cree que son similares a las características que permitieron la evolución de la vida en la Tierra. La localización y muestreo de estos manantiales podría ofrecer una visión más profunda de la biosfera profunda de la Tierra y de la historia de la vida en la Tierra. Tradicionalmente, los manantiales se han localizado mediante técnicas de campo costosas y lentas. El trabajo de campo puede ser peligroso. El objetivo de este estudio era desarrollar un modelo que permitiera localizar estas características geológicas únicas sin tener que ir primero al campo, ahorrando así tiempo, dinero y reduciendo los riesgos asociados a las localidades remotas sobre el terreno. Un análisis de idoneidad de yacimientos SIG funciona superponiendo datos georreferenciados existentes en un programa informático y sumando los distintos conjuntos de datos tras asignar un valor numérico a los campos importantes. Para este proyecto, he utilizado mapas de aguas superficiales y subterráneas, mapas geológicos, un mapa de suelos y un mapa de fallas de cuatro condados del norte de California. El modelo ha demostrado que es posible utilizar este tiempo de modelo y aplicarlo a un área geológica compleja para producir un mapa de campo utilizable para futuros trabajos de campo.

\chapter*{Abstract}
GIS is a powerful tool that can be used to locate springs sourced in ophiolites. The unique features associated with these springs include a reducing subsurface environment reacting at low temperatures producing high pH, Ca-rich formation fluids with high dissolved hydrogen and methane. Because of their unique chemical characteristics, these areas are often associated with microbes and are thought to be similar to the features that enabled life to evolve on Earth. Locating and sampling these springs could offer a deeper look into Earth's deep biosphere and the history of life on Earth. Springs have tradiitionally been located using expensive and time consuming field techniques. Field work can be dangerous. The goal of this study was to develop a model that could locate these unique geological features without first going into the field, thus saving time, money and reducing the risks associated with remote field localities. A GIS site suitability analysis works by overlaying existing geo-referenced data into a computer program and adding the different data sets after assigning a numerical value to the important fields. For this project, I used surface and ground water maps, geologic maps, a soil map, and a fault map for four counties in Northern California. The model has demonstrated that it is possible to use this time of model and apply it to a complex geologic area to produce a usable field map for future field work.

\chapter*{Agradecimientos}
First and foremost I am extremely grateful to my supervisors, Prof. XXX and Dr. XXX for their invaluable advice, continuous support, and patience during my PhD study. Their immense knowledge and plentiful experience have encouraged me in all the time of my academic research and daily life. I would also like to thank Dr. XXX and Dr. XXX for their technical support on my study. I would like to thank all the members in the XXX. It is their kind help and support that have made my study and life in the UK a wonderful time. Finally, I would like to express my gratitude to my parents, my wife and my children. Without their tremendous understanding and encouragement in the past few years, it would be impossible for me to complete my study.

\tableofcontents

\chapter{Introducción}
\section{Contexto}

En los últimos años, tras el reciente periodo de pandemia y la situación económica del país, ha quedado patente la necesidad de una buena gestión sanitaria por parte de la administración con el fin de atender la creciente demanda de servicios asistenciales por parte de la población haciendo un uso eficiente de los recursos sanitarios.

La correcta gestión de los recursos en la sanidad es un tema que en menor o mayor grado concierne a todos los ciudadanos por un motivo principal: el derecho a la protección de la salud. Este se reconoce en el artículo 43 de la Constitución y se concreta en la Ley General de Sanidad (Ley 14/1986), que establece su financiación pública, universalidad y gratuidad; su descentralización autonómica y su integración en el Sistema Nacional de Salud (SNS). En definitiva, todas las personas tienen derecho a una atención sanitaria de calidad en condiciones de igualdad.

En base a esta premisa, recientemente se han sucedido numerosas manifestaciones en defensa de la sanidad pública por parte de distintas asociaciones y representantes que proponen un aumento de las partidas destinas a equipos y personal del sistema sanitario público. Este trabajo se plantea desde otra punto de vista

% El Hospital Universitario Río Hortega de Valladolid, como su propio nombre indica, es un hospital universitario, en el que se da la oportunidad a todos los alumnos de la Universidad de Valladolid, independientemente de la rama a la que pertenezcan, a formarse y aprender en el ámbito de la salud donde se integran no sólo las ciencias de la salud sino también aquellas ramas del conocimiento referidas a técnicas sanitarias como la informática de la salud, técnicas de provisión de recursos sanitarios u otras muchas destrezas.
% El presente proyecto se ha realizado bajo un convenio de cooperación educativa entre la Universidad de Valladolid y el Hospital Universitario del Río Hortega en su unidad de procesos, con la premisa de reforzar su carácter universitario y conseguir una organización que genere conocimiento a través de su labor investigadora, que sea abierta a la sociedad, que utilice eficientemente los recursos y que sea sostenible, responsable y respetuosa con el medio ambiente.

\section{Motivación personal}

El sector sanitario es un campo en el que no es común encontrar a un ingeniero industrial. Y cuando esto sucede, normalmente se suele asociar a aspectos técnicos como pueden ser el mantenimiento de las instalaciones térmicas y eléctricas o la ingeniería biomédica.

\section{Objetivos}

\section{Alcance}

% El área de estudio comprende una unidad de hospitalización concreta en la que el sistema de doble cajón o “Kanban” se ha implantado de forma experimental en productos farmacéuticos, para en un futuro si se cumplen los objetivos citados en el punto 1.3 de este trabajo, extender su implantación y funcionamiento al resto de unidades.

% El proyecto consta de varias partes, en primer lugar, el estudio previo del consumo en el año 2021 del botiquín farmacéutico de la unidad de hospitalización de Medicina Interna nos permitirá establecer un pacto basado en el consumo real. Para ello, se deberá comprender el funcionamiento del Sistema Kanban, ya incorporado para el material fungible y que ahora da paso a la creación de este mismo sistema para el botiquín farmacéutico.

% Una segunda parte será el planteamiento del catálogo de artículos implicados en el estudio, cantidades y ubicaciones con ayuda del personal de enfermería involucrado; este catálogo se distribuye en los armarios y habitaciones prestadas al estudio, se adecua el mobiliario en términos de una mayor accesibilidad, se etiquetan todos los artículos e incluyen en el programa informático del HURH y la supervisora de la unidad se cerciora de que el personal de enfermería implicado conoce el funcionamiento de este nuevo sistema.

% Por último, su puesta en marcha, el personal oportuno deberá asegurarse de que las etiquetas se encuentran debidamente colocadas en el panel antes de primera hora de la mañana, que es cuando se hará la lectura de los pedidos y el departamento de logística empezará a tramitarlo. El mismo día que se comience, se iniciará un estudio sobre los consumos de las dos unidades con los catálogos planteados.

\section{Estructura memoria}

Esta memoria sigue la estructura especificada en la guía docente para la asignatura TFM del Máster en Ingeniería Industrial de la Escuela de Ingenieros Industriales (Universidad de Valladolid). \cite{lion2010}


% \section{Perfil de la organización}
% La Gerencia de Asistencia Sanitaria de Valladolid Oeste presta sus servicios a la población situada geográficamente en la zona oeste de la provincia, aproximadamente la mitad de Valladolid.

% La asistencia sanitaria se categoriza en dos niveles: Atención Especializada y Atención Primaria.

\chapter{Sistema sanitario en Castilla y León}
En el siguiente capítulo se detalla la estructura organizativa del sistema sanitario de Castilla y León a lo largo de todos los estratos desde la conserjería de sanidad hasta el centro de salud de Laguna de Duero-

\section{Perfil de la organización}

El Área de Salud Valladolid Oeste (ASVAO) presta asistencia sanitaria a la población de aproximadamente la mitad de la provincia de Valladolid, la situada geográficamente en la zona oeste. Esta atención sanitaria se provee en sus dos niveles asistenciales: Atención Primaria (con 17 Zonas Básicas de Salud) y Atención Hospitalaria (con el Hospital Universitario Rio Hortega y el Centro de Especialidades de Arturo Eyries).

Esta organización es fruto del trabajo integrado de las Gerencias de Atención Primaria y Hospitalaria.

El Área de Salud Valladolid Oeste (ASVAO) forma parte del Sistema Público de Salud de Castilla y León, el cual comprende el conjunto de actuaciones y recursos públicos de la Administración Sanitaria de la Comunidad Autónoma y de las Corporaciones Locales, cuya finalidad es la promoción y protección de la salud en todos sus ámbitos, la prevención de la enfermedad, la asistencia sanitaria y la rehabilitación, todo ello bajo una perspectiva de asistencia sanitaria integral (Ley 8/2010, de 30 de agosto, de ordenación del sistema de salud de Castilla y León).

El Hospital Universitario Río Hortega fue inaugurado como Centro Hospitalario el 24 de Julio de 1953. En ese momento disponía de 310 camas y 72 cunas. Tras más de 50 años ubicado en pleno centro (cerca de la plaza de San Pablo), se trasladó a un nuevo edificio en el año 2008, esta vez situado a las afueras de la ciudad, en el barrio de las Delicias. En estos momentos cuenta con 640 camas instaladas y se configura como un hospital general, universitario y de tercer nivel. Además de prestar atención sanitaria al Área de Salud Valladolid Oeste, para determinadas prestaciones actúa como “Servicio de Referencia”, ampliando su cobertura a toda la provincia de Valladolid o incluso a varias provincias limítrofes. En algunos casos presta servicio a toda la Comunidad Autónoma.

En BOCYL de 24 de octubre de 2008, según ACUERDO 111/2008 de 23 de octubre de la Junta de Castilla y León, se reestructura el Área de Salud de Valladolid Oeste y el Área de Salud de Valladolid Este.

La apertura del nuevo Hospital Universitario Río Hortega, que cambia de ubicación, hace necesaria la adaptación del mapa sanitario del Área de Salud de Valladolid Oeste, pasando las Z.B.S. Centro Gamazo, Z.B.S. la Victoria, Z.B.S. Rural I, Z.B.S. Cigales a pertenecer al Área Este de Valladolid y las Z.B.S. Delicias I y Z.B .S. Delicias II pasan a pertenecer al Área Oeste de Valladolid.

Al frente de nuestra organización se han sucedido los siguientes nombramientos en los últimos años: Por Orden SAN/465/2012 de 18 de junio se nombró Gerente de Atención Especializada a D. Alfonso Montero Moreno, posteriormente por resolución de 15 de febrero de 2012 del Gerente Regional de Salud se acumularon las funciones de Gerente de Atención Primaria a las de Gerente de Atención Hospitalaria. El 18 de julio de 2015 fue nombrado como Gerente de Atención Primaria D. Eduardo García Prieto. Por último en el año 2018 a través de Orden SAN/901/2018 de 10 de agosto se nombró Gerente de Atención Especializada a D. José Miguel García Vela.

A partir del mes de marzo de 2017 todo el personal de la Gerencia de Atención Primaria se integró a trabajar en las instalaciones del HURH junto a de los diferentes servicios correspondientes.

\section{Cartera de servicios}

El Área de Salud Valladolid Oeste (ASVAO) es la organización a la que se adscribe todo el dispositivo sanitario del Área, compuesto por los 17 Centros de A. Primaria y por el Hospital Universitario Río Hortega. El objetivo primordial de esta organización es la prestación de asistencia sanitaria a la población del Área Oeste de Valladolid, tanto en Atención Primaria, enAtención Hospitalaria como en Atención Sociosanitaria (esta última compartida con la Gerencia Regional de Servicios Sociales).

La Atención Primaria es el nivel básico e inicial de atención, que garantiza la globalidad y continuidad de ésta a lo largo de la vida del paciente, actuando como gestor y coordinador de casos y regulador de flujos. Comprende actividades de promoción de la salud, educación sanitaria, prevención de la enfermedad, asistencia sanitaria, mantenimiento y recuperación de la salud, así como la rehabilitación física y el trabajo social.

La Atención Hospitalaria comprende las actividades asistenciales, diagnósticas, terapéuticas y de rehabilitación y cuidados, así como aquellas de promoción de la salud, educación sanitaria y prevención de la enfermedad, cuya naturaleza aconseja que se realicen en ese nivel. La Atención Hospitalaria garantizará la continuidad de la atención integral al paciente, una vez superadas las posibilidades de la Atención Primaria y hasta que aquel pueda reintegrarse en dicho nivel. Prestará además, servicios de hospitalización en régimen de internamiento, asistencia especializada en consultas, hospital de día (médico y quirúrgico), atención paliativa a enfermos terminales, salud mental y rehabilitación en pacientes con déficit funcional recuperable.

La Atención Sociosanitaria comprende el conjunto de cuidados destinados a aquellos enfermos, generalmente crónicos, que por sus especiales características pueden beneficiarse de la actuación simultánea y sinérgica de los servicios sanitarios y sociales para aumentar su autonomía, paliar sus limitaciones o sufrimientos y facilitar su reinserción social. En el ámbito sanitario, la Atención Sociosanitaria comprenderá los cuidados sanitarios de larga duración, la atención sanitaria a la convalecencia y la rehabilitación en pacientes con déficit funcional recuperable.

\subsection{Servicios en Atención Primaria}

La cartera de servicios en Atención Primaria de Valladolid Oeste está compuesta por:

\begin{table}[H]
    \centering
    \begin{tabular}{l}
        \toprule
        Servicios                                             \\
        \midrule
        Medicina familiar y comunitaria                       \\
        Pediatría                                             \\
        Enfermería                                            \\
        Unidad de salud bucodental                            \\
        Unidad de atención a la mujer                         \\
        Fisioterapia                                          \\
        Extracciones laboratorio                              \\
        Radiología                                            \\
        Urgencias                                             \\
        Asistemcia social                                     \\
        Cirugía menor                                         \\
        Diagnóstico ecográfico                                \\
        Farmacia                                              \\
        Unidad administrativa de citas y atención al paciente \\
        Prevención y promoción de la salud                    \\
        \bottomrule
    \end{tabular}
    \caption{Cartera de servicios en atención primaria}
\end{table}

Existe un amplio capítulo de actuaciones dirigidas a la Prevención y Promoción de la Salud, en el que se incluyen, entre otros, programas de vacunación infantil, programas de vacunación en el adulto, desarrollo de actividades preventivas en el adulto sano, prevención de obesidad infantil, atención a pacientes crónicos (hipertensión arterial, dislipemias, diabetes, EPOC, etc.), atención al paciente crónico pluripatológico complejo, prevención precoz de cáncer de mama, de colon, deshabituación tabáquica, atención a pacientes ancianos de riesgo o en situación terminal, violencia de género, etc.

\subsection{Servicios en Atención Especializada}

En la siguiente tabla se refleja la Cartera de Servicios Básica del Hospital Universitario Río Hortega.

\section{Mercado servido}

% Insertar pirámide de población actualizada

\section{Zonas Básicas de Salud}

Una Zona Básica de Salud está formada por un conjunto de profesionales sanitarios y no sanitarios, que constituyen el equipo de Atención Primaria, y que son los responsables de la prestación de la atención de la salud a la población en su demarcación sanitaria, y ello, de forma coordinada, integral, permanente y continuada, y orientada al individuo, a la comunidad y al medio ambiente.

Entre los integrantes del equipo se encuentran médicos de familia, pediatras, enfermeros, matronas, auxiliares de enfermería, trabajadores sociales, auxiliares administrativos y celadores. Además, integrando funcionalmente, existen veterinarios y farmacéuticos.

Cada zona de salud dispone de un centro de salud, estructura dotada de los medios necesarios para la prestación de las funciones que debe desempeñar el equipo de atención primaria. Además, del centro de salud, existen consultorios locales destinados a aquellas localidades de más de 50 habitantes, donde los profesionales sanitarios atienden la demanda asistencial bajo el criterio de una correcta accesibilidad de los servicios a la población.

El equipo de atención primaria se organiza jerárquicamente bajo la supervisión del coordinador del centro de salud, nombrado de entre los facultativos del equipo, responsable de la gestión de los recursos humanos y materiales. Entre sus funciones se encuentran las siguientes:

\begin{itemize}
    \item Asumir la representación oficial del equipo
    \item Ejercer la dirección y coordinación de todo el personal en lo referente al régimen interno de funcionamiento, así como resolver los conflictos en lo refenrente a dicho régimen interno de funcionamiento, y estimular el trabajo en equipo.
    \item Participar en la gestión económica del centro.
    \item Coordinar, supervisar y controlar las diversas actividades desarrolladas en la zona.
    \item Presidir el consejo de salud de la zona básica de salud.
\end{itemize}

Por otra parte, el coordinador del equipo cuenta con la colaboración de un responsable de enfermería, con funciones de su supervisión y coordinación de las actividades de los profesionales de enfermería y unos responsables de programas de las áreas funcionales del equipo que son:

\begin{itemize}
    \item Área de Atención Directa
    \item Área de Docencia e Investigación
    \item Área de Administración
\end{itemize}

Las funciones del equipo de atención primaria vienen recogidas en la normativa que regula la organización funcional de las zonas de salud, entre las que cabe destacar:

\begin{itemize}
    \item Funciones de Salud Pública
    \item Funciones de Asistencia Sanitaria
    \item Funciones Docentes
    \item Funciones de Investigación
    \item Funciones Administrativas
    \item Funciones de Participación Comunitaria
\end{itemize}

Respecto a esta última, en el ámbito de cada Zona Básica de Salud se encuentra constituido el llamado Consejo de Salud, cuyo presidente es el propio Coordinador y donde están representados: Alcaldes, Asociaciones de Vecinos, Asociaciones de Consumidores, Ministerio de Educación y Ciencia, Sindicatos y otras asociaciones o grupos de ciudadanos legalmente constituidos y con fines de promoción comunitaria. De esta manera, el Consejo de Salud tiene la consideración de «Órgano de Participación Comunitaria» en las tareas de salud de cada Zona Básica.

Por otra parte, se contempla la existencia de algunos profesionales de Área en Atención Primaria para apoyar y complementar la labor de los Equipos de Atención Primaria.

Por otra parte, se contempla la existencia de algunos profesionales de Área en Atención Primaria para apoyar y complementar la labor de los Equipos de Atención Primaria. En este sentido, se han creado una serie de Unidades de Área que, por motivos de eficiencia, trabajan en más de una Zona Básica de Salud y son diferentes en cada zona, en función de criterios demográficos y demandas asistenciales. Las Unidades de Área son las siguientes.

\begin{itemize}
    \item Unidades de Fisioterapia
    \item Unidades de Salud Bucodental
    \item Unidades de Matronas
    \item Unidades de Pediatría de Área
    \item Unidades de Atención Urgente (PAC)
    \item Unidades de Atención a Domicilio (ESAD)
\end{itemize}

La oferta de servicios de Atención Primaria esta recogida en la cartera de servicios e incluye:

\begin{itemize}
    \item Servicios de atención al niño (vacunaciones infantiles, revisión del niño sano, prevención de la caries infantil y educación sanitaria en la escuela)
    \item Servicios de atención a la mujer (captación y seguimiento del emmabarazo, preparación al parto y visita posparto, vacunación de la rubeola, anticoncepción, prevención del cáncer ginecológico y atención en el climaterio)
    \item Servicios de atención al adulto-anciano (vacunaciones del adulto: gripe, tétanos, hepatitis B a grupos de riesgo; prevención de enfermedades cardiovasculares, atención a enfermos crónicos, atención domiciliaria a inmovilizados y terminales, prevención y detección de problemas en el anciano)
    \item Tratamientos fisioterapéuticos básicos y cirujía menor
\end{itemize}

Desde los equipos de atención primaria se presta atención sanitaria urgente las 24 horas del día, disponiendo para ello, especialmente en el medio urbano, de servicios de urgencia de refuerzo.

La implantación del modelo de Atención Primaria ha supuesto la modernización de los dispositivos del primer nivel asistencial (construcción de centros de salud y consultorios locales, mejora del equipamiento y de la tecnología médica, formación postgraduada y formación continuada de los médicos de familia, incorporación de profesionales sanitarios que refuerzan la oferta asistencial en Atención Primaria)

Ha supuesto también la integración de actividades de promoción y prevención de la salud y de cuidados de enfermería, la utilización sistemática de registros clínicos y el acceso a tecnología médicas.

En todos los centros de salud que atienden a una población superior a 10000 habitantes, existe una Unidad de Atención al Usuario, responsable del sistema de cita previa telefónica para el acceso a los servicios asistenciales, gestión de reclamaciones y sugerencias, gestión de prestaciones e información de usuario.

El Sistema de Gestión de Atención Primaria está basado en una estrategia de descentralización de las funciones de financiación y de provisión de servicios, según el cual las Áreas de Atención Primaria asumen la responsabilidad de la gestión de recursos y de los centros. Estos, a su vez, acuerdan con los equipos directivos de cada área un pacto de objetivos anual que incluye: cobertura de los servicios, cumplimiento de normas técnicas o criterios de calidad científico-técnica, objetivos de prescripción farmacológica, etc. El pacto de gestión incluye, igualmente, los presupuestos asignados al equipo para gastos de personal, farmacia, formación continuada de los profesionales además de compras y equipamiento.

\chapter{Herramientas}
\chapter{Desarrollo}
\chapter{Conlusiones}
\chapter*{Bibliografía}

\appendix
\chapter{Anexos}

\printbibliography

\end{document}